\chapter{Mekanikk}
Reaksjoner i systemet:

For å utvikle ideer til hvordan problemstillingen kan løses, er det nødvendig å analysere hva som skjer om en hjulmutter løsner. Det har i denne sammenheng blitt brukt en brainstorming-teknikk for å sammen kunne komme fram til så mange av disse reaksjonene som mulig. Vi har identifisert følgende reaksjoner grunnet endring i mutterens tilstramming:

\begin{table}[h]
\begin{tabular}{|l|l|}
\hline
\textbf{Mekanisk}                   & \textbf{Annet}                                   \\
\hline
Strekk i bolt ved strammet mutter   & Temperaturendring pga friksjonsvarme ved løsning \\
\hline
Kompresjonsendring i mutter         & Vibrasjonsendringer i understell                 \\
\hline
Torsjonsendring                     & Endring i friksjon mellom bakke og hjul          \\
\hline
Trykkendring mellom mutter og flate & ...                                              \\
\hline
...                                 &                                                 \\
\hline
\end{tabular}
\end{table}

Etter å ha hørt med Olav Rogne, yrekssjåfør, har vi valgt å bruke dimensjonene han har oppgitt som utregningseksempel. Det ble opplyst om at det var normalt å bruke mutter av typen M33x3.5 som trekkes til med et tiltrekkingsmoment på 700Nm. Man kan utifra dette enkelt regne ut spenninger som oppstår i bolten.

Momentet som tilføres bolten ved tiltrekking deles opp i to komponenter:

$M\_{tiltrekning}=M\_{v}+M\_{s}$

Der $M\_{v}$ er momentet som oppstår pga friksjonskrefter i gjengeflaten, og $M\_{s}$ er momentet som oppstår pga krefter mellom mutter og underlaget. Disse kan igjen bli beskrevet slik:

$M\_{v}=F\cdot tan(\varphi +\varepsilon )\cdot r\_{m}$

og

$M\_{s}=\mu 'Fr\_{m}$

I disse to formlene tar vi i bruk flere fraktorer.  r\_{m} er radiusen som friksjonskraften mot underlaget antas å virke på:

$r\_{m}=\frac{N\o kkelvidde+d\_{h}}{4}$

$\varphi$ er gjengenes stigningsvinkel:

$\varphi =\frac{gjengestigning}{\pi \cdot d\_{pich}}$

og $\varepsilon$ er friksjonsvinkelen i skråplanet der kreftene i gjengene virker:

$\varepsilon =tan^{-1}(\frac{\mu}{cos(\alpha )})$

Her er $\alpha$ halve gjengevinkelen og vi har satt friksjonsfaktor i gjengeene til $\mu=0.08$ \cite{FriksjonsfaktorGjenger} %http://www.kamax.com/fileadmin/user\_upload/dokumente/pdf/Bolt\_and\_Screw\_Compendium.pdf

Videre kan vi løse dette for kraften:

$F\_{aksiell}=\frac{M\_{tiltrekning}}{(tan(\varphi +tan^{-1}(\frac{\mu }{cos(\alpha )}))+\mu')\frac{N+d\_{h}}{4}}$

Siden vi også vet at

$\sigma=\frac{F}{A}$

Kan vi si at trykket mellom mutter og underlag er:

$p\_{trykkflate}=\frac{F\_{aksiell}}{A\_{trykkflate}}=\frac{F\_{aksiell}}{\frac{\pi }{4}(D\_{mutterflate}^{2}-D\_{hull}^{2})}$

Videre vet vi at flytegrensen er:


$\sigma \_{f}\approx 250 Mpa-350 Mpa$
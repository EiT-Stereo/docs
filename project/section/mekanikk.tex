\section{Mekanikk}
Reaksjoner i systemet:

For å utvikle ideer til hvordan problemstillingen kan løses, er det nødvendig å analysere hva som skjer om en hjulmutter løsner. Det har i denne sammenheng blitt brukt en brainstorming-teknikk for å sammen kunne komme fram til så mange av disse reaksjonene som mulig. Vi har identifisert følgende reaksjoner grunnet endring i mutterens tilstramming:

\begin{table}[h]
\caption{\comment{TODO: skrive noe her}}
\begin{tabular}{|l|l|}
\hline
\textbf{Mekanisk}                   & \textbf{Annet}                                   \\
\hline
Strekk i bolt ved strammet mutter   & Temperaturendring pga friksjonsvarme ved løsning \\
\hline
Kompresjonsendring i mutter         & Vibrasjonsendringer i understell                 \\
\hline
Torsjonsendring                     & Endring i friksjon mellom bakke og hjul          \\
\hline
Trykkendring mellom mutter og flate & ...                                              \\
\hline
...                                 &                                                 \\
\hline
\end{tabular}
\end{table}

Etter å ha hørt med Olav Rogne, yrekssjåfør, har vi valgt å bruke dimensjonene han har oppgitt som utregningseksempel. Det ble opplyst om at det var normalt å bruke mutter av typen M33x3.5 som trekkes til med et tiltrekkingsmoment på 700Nm. Man kan utifra dette enkelt regne ut spenninger som oppstår i bolten.

Momentet som tilføres bolten ved tiltrekking deles opp i to komponenter:
\begin{center}
$M_{tiltrekning}=M_{v}+M_{s}$
\end{center}
Der $M_{v}$ er momentet som oppstår pga friksjonskrefter i gjengeflaten, og $M_{s}$ er momentet som oppstår pga krefter mellom mutter og underlaget. Disse kan igjen bli beskrevet slik:
\begin{center}
$M_{v}=F\cdot tan(\varphi +\varepsilon )\cdot r_{m}$
\end{center}
og
\begin{center}
$M_{s}=\mu 'Fr_{m}$
\end{center}
I disse to formlene tar vi i bruk flere fraktorer. $ r_{m}$ er radiusen som friksjonskraften mot underlaget antas å virke på:
\begin{center}
$r_{m}=\frac{N\o kkelvidde+d_{h}}{4}$
\end{center}
$\varphi$ er gjengenes stigningsvinkel:
\begin{center}
$\varphi =\frac{gjengestigning}{\pi \cdot d_{pich}}$
\end{center}
og $\varepsilon$ er friksjonsvinkelen i skråplanet der kreftene i gjengene virker:
\begin{center}
$\varepsilon =tan^{-1}(\frac{\mu}{cos(\alpha )})$
\end{center}
Her er $\alpha$ halve gjengevinkelen og vi har satt friksjonsfaktor i gjengeene til $\mu=0.08$ \cite{FriksjonsfaktorGjenger} %http://www.kamax.com/fileadmin/user_upload/dokumente/pdf/Bolt_and_Screw_Compendium.pdf

Videre kan vi løse dette for kraften:
\begin{center}
$F_{aksiell}=\frac{M_{tiltrekning}}{(tan(\varphi +tan^{-1}(\frac{\mu }{cos(\alpha )}))+\mu')\frac{N+d_{h}}{4}}$
\end{center}
Siden vi også vet at
\begin{center}
$\sigma=\frac{F}{A}$
\end{center}
Kan vi si at trykket mellom mutter og underlag er:
\begin{center}
$p_{trykkflate}=\frac{F_{aksiell}}{A_{trykkflate}}=\frac{F_{aksiell}}{\frac{\pi }{4}(D_{mutterflate}^{2}-D_{hull}^{2})}$
\end{center}
Videre vet vi at flytegrensen er:

\begin{center}
$\sigma _{f}\approx 250 Mpa-350 Mpa$
\end{center}
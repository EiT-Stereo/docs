
Reaksjoner i systemet:

For å utvikle ideer til hvordan problemstillingen kan løses, er det nødvendig å analysere hva som skjer om en hjulmutter løsner. Det har i denne sammenheng blitt brukt en brainstorming-teknikk for å sammen kunne komme fram til så mange av disse reaksjonene som mulig. Vi har identifisert følgende reaksjoner grunnet endring i mutterens tilstramming:

\begin{table}[h]
\begin{tabular}{ll}
\textbf{Mekanisk}                   & \textbf{Annet}                                   \\
Strekk i bolt ved strammet mutter   & Temperaturendring pga friksjonsvarme ved løsning \\
Kompresjonsendring i mutter         & Vibrasjonsendringer i understell                 \\
Torsjonsendring                     & Endring i friksjon mellom bakke og hjul          \\
Trykkendring mellom mutter og flate & ...                                              \\
...                                 &                                                 
\end{tabular}
\end{table}

Etter å ha hørt med “lastebilsjåfør” har vi valgt å bruke dimensjonene han har oppgitt som utregningseksempel. Det ble opplyst om at det var normalt å bruke mutter av typen M33x3.5 som trekkes til med et tiltrekkingsmoment på 700Nm. Man kan utifra dette enkelt regne ut spenninger som oppstår i bolten.

Momentet som tilføres bolten ved tiltrekking deles opp i to komponenter:

M_{tiltrekning}=M_{v}+M_{s}

Der M_{v} er momentet som oppstår pga friksjonskrefter i gjengeflaten, og M_{s} er momentet som oppstår pga krefter mellom mutter og underlaget. Disse kan igjen bli beskrevet slik:

M_{v}=F\cdot tan(\varphi +\varepsilon )\cdot r_{m}

og

M_{s}=\mu 'Fr_{m}

I disse to formlene tar vi i bruk flere fraktorer.  r_{m} er radiusen som friksjonskraften mot underlaget antas å virke på:

r_{m}=\frac{N\o kkelvidde+d_{h}}{4}

\varphi er gjengenes stigningsvinkel:

\varphi =\frac{gjengestigning}{\pi \cdot d_{pich}}

og \varepsilon er friksjonsvinkelen i skråplanet der kreftene i gjengene virker:

\varepsilon =tan^{-1}(\frac{\mu}{cos(\alpha )})

Her er \alpha halve gjengevinkelen og vi har satt friksjonsfaktor i gjengeene til \mu=0.08 %http://www.kamax.com/fileadmin/user_upload/dokumente/pdf/Bolt_and_Screw_Compendium.pdf

Videre kan vi løse dette for kraften:

F_{aksiell}=\frac{M_{tiltrekning}}{(tan(\varphi +tan^{-1}(\frac{\mu }{cos(\alpha )}))+\mu')\frac{N+d_{h}}{4}}

Siden vi også vet at

\sigma=\frac{F}{A}

Kan vi si at trykket mellom mutter og underlag er:

p_{trykkflate}=\frac{F_{aksiell}}{A_{trykkflate}}=\frac{F_{aksiell}}{\frac{\pi }{4}(D_{mutterflate}^{2}-D_{hull}^{2})}

Videre vet vi at flytegrensen er:


\sigma _{f}\approx 250 Mpa-350 Mpa
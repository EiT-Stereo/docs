\chapter{Introduksjon}
\section{Gruppen}
Gruppen består av seks studenter fra Norges Teknisk-Naturvitenskapelige
Universitet, som tar kurset TDT4856 - Eksperter i Team. Gruppemedlemmene
har alle forskjellig teknologisk bakgrunn.
\begin{description} %TODO Bilder av hver enkelt over beskrivelsen
	\item[Erlend Aksnes] Erlend studerer fjerde året informatikk og har erfaring med Java, JavaScript, PowerShell og SQL. I tillegg har han erfaring med servere og nettverk etter bachelor i informatikk (spes. drift av datasystemer) ved HiST. 
	\item[Emil Dale Bjølykhaug] %TODO About
	\item[Christoffer Ramstad-Evensen] \comment{\hfill \\} Studerer fjerde
  året elektronikk ved NTNU. Han har erfaring innen krets design
  og lavnivå programmering av embedded systems. Han har fokus på energieffektive
  systemer og optimalisering. Teamarbeidserfaring har han fra lederstilling i
  linjeforeringer og sommerjobb hos ARM. Han er tålmodig og interessert i
  teknologi.
	\item[David Johän Hovind] %TODO About
	\item[Kristoffer Løvall] \comment{\hfill \\} Studerer fjerde året maskin ved NTNU. 
	Han har kompetanse innen produksjon, design og materialer, og erfaring
	fra hobbyer som tegning og konstruksjon av modellfly, racing og vedlikehold
	av go-kart, m.m. Han har erfaring med teamarbeid gjennom diverse verv som fag-
	og forskningsansvarlig i Maskiningeniørenes Linjeforening (MILF ved HiB), samt
	ledererfaring gjennom flere prosjekter og fag gjennom utdannelsen.
	
	\item[Nina Margrethe Smørsgård] \comment{\hfill \\}Studerer fjerde året
	informatikk. Hun har erfaring med både web- og embedded programmering, med
	teknologier som Android, Arduino, C, Java, JavaScript og Python. Hun har
	erfaring med teamarbeid fra sommerinternships i Finn.no, Iterate og DiFi,
	og andre prosjektarbeider på NTNU, samt ledererfaring fra NTNUi.
\end{description}

\section{Initiativet}
Ideen for dette prosjektet sprang ut fra kurset Eksperter i Team på NTNU, med 
temaet ``IT-styring av moderne biler''.\cite{tdt4856} Fokuset i dette temaet lå på lastebil 
og tyngre kjøretøy, og Kongsberg Automotive sto som bistående kunde.

Aktuelle nyheter i oppstartsperioden av dette prosjektet rapporterte om en 
hendelse i Kina, hvor et løpsk lastebilhjul traff en mann som hadde parkert på 
veiskulderen, og dyttet ham over autovernet og ned en skrent. Heldigvis pådro 
han seg ingen større skader i ulykken. Denne hendelsen inspirerte til en studie 
av frekvensen av denne type ulykker, og om løse lastebilhjul er et problem 
verdt å løse.

\section{Ideen}
Ideen er å utvikle et produkt som rapporterer sanntidsstatus til lastebilsjåføren 
ved løsnede hjulmuttre. Dette innebærer en måling av hjulmuttrenes tilstand på 
hjulbolten, analyse av disse dataene, og rapportering av nødvendige data 
opp til førerhuset slik at de kan visualiseres på et lett forståelig format.

Den første mulige løsningen som ble oppdaget var måling av mutterposisjon ved 
hjelp av strekklapper. Etter konsultasjon med Kongsberg Automotive ble det klart 
at denne typen avviksregistrering ville egnet seg best for ettermontering, 
men at det allikevel ville bli et dyrt produkt basert på antall nødvendige 
strekklapper pr. hjul. 

Løsningsalternativ nummer to var en vibrasjonssenor i nav eller aksling, som 
rapporterer tilbake dersom det registreres avvik i vibrasjoner fra ett hjul 
kontra de andre hjulene på kjøretøyet. En slik løsning vil kunne bli billigere 
basert på et lavere behov for komponenter.

Feilmeldingen som kommer opp til sjåfør presenteres så enkelt som mulig for å unngå for mange forstyrrende elementer i dashbordet. Samtidig er nøyaktighet en viktig del av representasjonen, slik at sjåføren sparer tid på å kun ettertrekke de nødvendige muttre.

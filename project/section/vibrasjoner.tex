\section{Vibrasjoner}
Vibrasjoner er bølger som propagerer gjennom et medium. Det finnes i 
all hovedsak tre typer bølger: tverrgående, langsgående og overflatebølger. 
Disse bølgene kan måles med et akselerometer.

\subsection{Vibrasjon av løst hjul}
Aller enkleste måte å måle om mutterene er løse er å vente til 
de er helt løse slik at hjulet også blir løst. Når felgen bare 
henger på boltene (i motsetning til friksjonen på anleggsflata 
mot navet) vil den forflytte seg ut av senter i takt med rotasjon 
og dermed skape en vibrasjon avhengig av rotasjonshastigheten. 
Denne måten å detektere løse bolter er langt i fra optimal da deler 
av skaden allerede har skjedd (skjærkrefter på bolter, kan klippes),
 selv om ``worst case scenario'' blir unngått (dekk flygende avgårde på veien).

\subsection{Vibrasjon av manglende mutter}
Det er mer ønskelig å kunne måle vibrasjoner før hjulet løsner totalt
fra navet. Noe som kanskje kan gjøre dette mulig er om en mutter som løsner
skrur seg helt av bolten før resten av mutterene har løsnet. Det roterende systemet vil
da få et tyngdepunkt som er ute av senter, og det blir dermed vibrasjoner.
Dette er derimot en noe upålitelig antagelse, og dersom mutterene er nøyaktig
likt strammet med tilsvarende friksjon på anleggsflater og gjenger vil de skru seg opp omtrent samtidig.
Det er heller ikke noe som sier at mutterene skal fortsette og skru seg ut, de kan like 
gjerne skru seg innover igjen (helt til de møter motstand i anleggsflater).

\subsection{Egenfrekvenser}
En mer lovende løsning er å måle endringer i egenfrekvensene til 
akslingene. En aksling med to faste hjul vil ha en annen egenfrekvens 
enn en aksling med et fast og et delvis løst hjul. Estimerte egenfrekvenser 
for en aksling kan beregnes ut fra FE modell. Disse egenfrekvensene må %TODO cite
da testes opp mot faktisk målinger på en aksling for å se om de stemmer. 
Et annet spørsmål som da oppstår angående egenfrekvenser er da om 
vibrasjoner pga. ujevnheter i veibanen vil være tilstrekkelig for å lage 
vibrasjoner med stor nok amplitude i akslingene. En må her ta hensyn til 
at dekket er laget av gummi og vil dermed fungere som et dempende ledd. 
Vibrasjon i akslingen kan eventuelt sikres med å plassere en innretning for 
å skape vibrasjoner, som en i tillegg til et akselerometer på akslingen.

\section{Vibrasjoner}
Vibrasjoner er bølger som propagerer gjennom et medium. Det finnes i 
all hovudsak tre typer bølger: tverrgående, langsgående og overflatebølger. 
Disse bølgene kan måles med et akselerometer.

\subsection{Vibrasjoner av løsnende mutter}
Aller enkleste måte å måle om mutterene er løse er å vente til 
de er helt løse slik at hjulet også blir løst. Når da felgen bare 
henger på boltene (i motsetning til friksjonen på anleggsflata 
mot navet) vil den forflytte seg ut av senter i takt med rotasjon 
og dermed skape en vibrasjon avhengig av rotasjonshastigheten. 
Denne måten å detektere løse bolter er langt i fra optimal da deler 
av skaden allerede har skjedd (skjærkrefter på bolter, kan klippes),
 selv om «worst case scenario» blir unngått (dekk flygende avgårde på veien).

En mer lovende løsning er å måle endringer i egenfrekvensene til 
akslingene. En aksling med to faste hjul vil ha en annen egenfrekvens 
enn en aksling med et fast og et delvis løst hjul. Estimerte egenfrekvenser 
for en aksling kan beregnes ut fra FE modell. Disse egenfrekvensene må 
da testes opp mot faktisk målinger på en aksling for å se om de stemmer. 
Et annet spørsmål som da oppstår angåande egenfrekvenser er da om 
vibrasjoner pga ujevnheter i veibanen vil være tilfredsstillende for å lage 
vibrasjoner med stor nok amplitude i akslingene. En må her ta hensyn til 
at dekket er laget av gummi og vil dermed fungere som et dempende ledd. 
Vibrasjon i akslingen kan eventuelt sikres med å plassere en høgtaler i 
tillegg til et akselerometer på akslingen.

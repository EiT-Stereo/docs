\section{Gruppen}
Gruppen består av seks studenter fra Norges Teknisk-Naturvitenskapelige
Universitet, som tar kurset TDT4856 - Eksperter i Team. Gruppemedlemmene
har alle forskjellig teknologisk bakgrunn.
\begin{description} %TODO Bilder av hver enkelt over beskrivelsen
	\item[Erlend Aksnes] Erlend studerer fjerde året informatikk og har erfaring med Java, JavaScript, PowerShell og SQL. I tillegg har han erfaring med servere og nettverk etter bachelor i informatikk (spes. drift av datasystemer) ved HiST. 
	\item[Emil Dale Bjølykhaug] Emil studerer første året på 2-årig master undervannsteknologi. 
Han har en bachelor innen Automasjon fra Hials og har kompetanse innen mikrokontrollere, PLS, 
kommunikasjon og styresystemer. I tillegg har han mekanisk kompetanse grunnet stor 
motorsportinteresse og fagbrev som industrimekaniker.
	\item[Christoffer Ramstad-Evensen] \comment{\hfill \\} Studerer fjerde
  året elektronikk ved NTNU. Han har erfaring innen krets design
  og lavnivå programmering av embedded systems. Han har fokus på energieffektive
  systemer og optimalisering. Teamarbeidserfaring har han fra lederstilling i
  linjeforeringer og sommerjobb hos ARM. Han er tålmodig og interessert i
  teknologi.
	\item[David Johän Hovind] %TODO About
	\item[Kristoffer Løvall] \comment{\hfill \\} Studerer fjerde året maskin ved NTNU. 
	Han har kompetanse innen produksjon, design og materialer, og erfaring
	fra hobbyer som tegning og konstruksjon av modellfly, racing og vedlikehold
	av go-kart, m.m. Han har erfaring med teamarbeid gjennom diverse verv som fag-
	og forskningsansvarlig i Maskiningeniørenes Linjeforening (MILF ved HiB), samt
	ledererfaring gjennom flere prosjekter og fag gjennom utdannelsen.
	
	\item[Nina Margrethe Smørsgård] \comment{\hfill \\}Studerer fjerde året
	informatikk. Hun har erfaring med både web- og embedded programmering, med
	teknologier som Android, Arduino, C, Java, JavaScript og Python. Hun har
	erfaring med teamarbeid fra sommerinternships i Finn.no, Iterate og DiFi,
	og andre prosjektarbeider på NTNU, samt ledererfaring fra NTNUi.
\end{description}
\section{Problemet}

Aktuelle nyheter i oppstartsperioden av dette prosjektet rapporterte om en 
hendelse i Kina, hvor et løpsk lastebilhjul traff en mann som hadde parkert på 
veiskulderen, og dyttet ham over autovernet og ned en skrent. Heldigvis pådro 
han seg ingen større skader i ulykken. Vi tok også kontakt med en lastebilsjåfør
 og maskiningeniør, Olav Rogne, for å forhøre oss om relevante problemstillinger 
til EiT. Hendelsen i Kina, samt intervjuet med yrkessjåføren inspirerte til videre
research av problemet, og følgende problemstilling ble formet:

\begin{center}
\emph{``Hvordan automatisere identifikasjon og varsling av løse hjumuttere?''}
\end{center}

Både på verdensbasis og i Norge er såkalt ``hjulmist'' et økende problem, ikke bare 
på grunn av materielle skader, men også på grunn av risikoen for menneskelige 
skader og dødsfall. Det finnes flere grunner til løsnede hjul, og gjennom undersøkelser 
gjort i flere land har man kommet fram til fire overordnede årsaker:

\begin{itemize}
\item{ Skader på friksjonsoverflaten mellom mutter, bolt, felg og nav fører til sluring og/eller
vibrasjoner som over tid gjør at bolten/mutteren løsner.}
\item { Temperaturendringer som medfører at bolt og mutter krymper eller vokser, noe som 
endrer spenningene og tiltrekkingsmomentet i koblingen.}
\item{ Manglende ettertrekking av hjulmuttere.}
\item{ Utmatting i bolt grunnet for hard tiltrekking over anbefalt tiltrekkingsmoment som igjen 
fører til strekk over materialets flytegrense (plastisk/varig deformasjon).}
\end{itemize}

For å unngå dette problemet ønsker vi å planlegge og utvikle en spec for et system som oppdager 
det løsnende hjulet før det separeres fra vogntoget. Vi ønsker å automatisere både identifikasjon
og varsling slik at sikkerheten både for sjåfør og andre blir drastisk bedret, i tillegg til at de materielle
kostnadene blir redusert.

%kilde http://www.egenbil.no/nyttig-aa-vite/hjulmist
\section{Markedsadgang}
Ettermontering av produktet vil være dyrere enn montering under produksjon av lastebil, man må koble fra akslingen for å montere vibrasjonssensor, samt trekke kabler og oppdatere software. Ettersom produktet vil bli rimeligst dersom det monteres under produksjon av lastebil, vil  samarbeid med en stor produsent, slik som feks. Volvo eller Scania være nødvendig for å entre markedet. 
\section{Størrelse på marked}
I de siste fire årene har det blitt produsert over 200 000 lasterbiler årlig i Europa\cite{lastebilprod-DAF}. Det vil si at markedet for produktet er veldig stort og vi har muligheten til å komme på markedet først. I tillegg er det mulighet for gjøre veien tryggere, da færre hjul vil løsne på lastebiler som har vårt produkt integrert.
\section{Konkurrenter}
Til nå er det ingen konkurrenter i markedet som har en digital løsning for automatisk deteksjon og varsling av løse hjulbolter. Det man kan finne på markedet i dage er manuelle løsninger, hvor man fester små indikatorer på boltene for å se om de har beveget seg, eller for å feste boltene bedre. De eksisterende løsningene kan man lese mer om i kapittel 2.5.
Vi ønsker å inngå samarbeid med store aktører for å oppnå en nisjerolle i markedet. På denne måten ser vi for oss å bli den foretrukne underleverandøren av automatisk deteksjon og varsling av løse hjulbolter.
\section{Kundens makt}
Selv om vi har fordelen med å være først ute på markedet med vår løsning vil det være flere å konkurrere med om kundene. Da vi er først ut i markedet vil vi raskt kunne oppnå en god dialog med kundene. Vi vil etablere oss som en fleksibel, løsningsorientert og pålitelig leverandør. Det stilles høye krav til komponenter som produseres til lastebiler, da komponentene skal være i drift gjennom en lang levetid. Vi vil derfor ha fokus på å levere produkter av høy kvalitet til en god pris. Vi skal ha et stort fokus på kvalitetssikring for å sikre at ingen produkter med feil havner hos sluttbruker. Vi ønsker fornøyde kunder og er derfor mottakelige for nye standarder og krav fra kundene. 
\section{Forhold til Leverandører}
Det er viktig at vi velger underleverandører som er anerkjent for å levere komponenter med høy kvalitet og pålitelighet. Samtidig må vi forholde oss til pris på komponenter fra underleverandører, da det vil være vanskelig å få solgt produktet til kunder om prisen er for høy. Vi skal kjøpe prøveeksemplarer fra forskjellige leverandører for å kunne sammenligne pris mot kvalitet. Vi skal sikre gode avtaler med underleverandørene, og binder en mindre del av vår kapital på lagerhold om vi har underleverandører som leverer komponenter på kort varsel. Vi forhandler frem langsiktige kontrakter for å sikre leveranse og pris, med forbehold om brudd ved mislighold av krav på kvalitet, pris og levering. 
\section{De fire P-ene}
\begin{description}
	\item[Produktstrategi] De to første årene ønsker vi å starte med utvikling av et enkelt produkt for 		automatisk deteksjon av løse hjulbolter. I år to og tre skal vi jobbe med videreutvikling av 				produktet. Det vil da være aktuelt å se på andre problemer som kan detekteres ved bruk av 			samme sensor. Med tid vil vi da kunne tilby et større produkt, og kunden vil kunne få mer 			funksjonalitet for pengene. Denne videreutviklingen vil fortsette etter de tre årene, helt til all aktuell 		funksjonalitet er implementert.
	\item[Prisstrategi] Delene som kreves for å lage produktet koster ca. 40 kr per aksling for å gjør integrering av produktet mulig. Ettersom vi ikke har innsikt i Kongsberg Automotive sin økonomi, vil prising av produktet være opp til salgsavdelingen deres. Etter hvert som bedriften skaffer seg egenkapital vil det være naturlig å gjøre enda større avtaler med underleverandører og kunder. Dette vil resultere i lavere kostnader ved innkjøp av komponenter og høyere fortjeneste ved salg.
	\item[Distribusjon og salgsstrategi] Vi planlegger å selge produktet til produsenter av lastebiler. De 		skal selge produktet til sine kunder som et tilleggsprodukt eller innbakt i prisen på lastebilen. 			Distribusjon vil derfor gå gjennom produsenter, og fokuset vårt vil være å inngå avtaler med 			produsenter. Vi skal ha et verdensomspennende marked og må derfor inngå avtaler med store 		produsenter slik som Daimler\cite{daimler}. Daimler er verdens største globale produsent av 				trailere over 6 tonn og produsererfor merkene Mercedez, Freightliner, Western Star, Fuso og 			BharatBenz.
	\item[Promoteringsstrategi] Vi skal ha stand på konferanser og messer hvor vi vil vise hvordan vårt 	produkt kan automatisk detetektere og varsle om løse hjulbolter. Her vil vi få kontakt med 			lastebilprodusenter og vise at vårt produkt vil tjene deres produkt. Promotering av produktet vil 		inngå i kundens promotering ut til sluttbruker, da produktet vil være en del av kundens produkt. 		Både kunder og 	sluttbruker vil ha tilgang til informasjon om våre produkter gjennom vår nettside.
\end{description}
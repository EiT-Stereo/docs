Det er i løpet av rapporten gått inn på to forskjellige framgangsmetoder for å detektere
løse hjulmutre. Det konkluderes med at den mest interessante løsningen er å analysere
vibrasjoner som oppstår i aksling når hjulmutrene løsner fremfor å måle strekk eller trykk i 
hver enkelt hjulmutter. Et av hovedargumentene for denne løsningen er kostnadseffektiviteten og 
muligheter for å kunne utvide funksjonaliteten til systemet. En slik løsning kan potensielt 
også detektere hjulets lufttrykk og bremseslitasje. I tillegg har det blitt lagt fokus på et 
brukervennlig grensesnitt for sjåfør. Systemet skal være lett å oppdatere og lett å integrere 
i det eksisterende systemet i en lastebil. 

En enkel markedsanalyse viser at konseptet er realiserbart til en rimelig pris og 
løsningen er etterspurt. 

For å kunne realisere konseptet kreves grundig testing og veien videre vil kreve
ressurser. Softwaren i systemet må optimaliseres for å kunne gjenkjenne de korrekte
vibrasjonene .
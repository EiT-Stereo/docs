Det er i løpet av rapporten gått inn på to forskjellige framgangsmetoder for å detektere
løse hjulmutre. Det konkluderes med at den mest interessante løsningen er å analysere
vibrasjoner som oppstår i aksling når hjulmutrene løsner fremfor å måle strekk eller trykk i 
hver enkelt hjulmutter. Et av hovedargumentene for denne løsningen er kostnadseffektiviteten og 
muligheter for å kunne utvide funksjonaliteten til systemet. En slik løsning kan potensielt 
også detektere hjulets lufttrykk og bremseslitasje. I denne rapporten blir det beskrevet et 
produkt som bruker vibrasjonssensorer for detekteringen av løse hjulmuttere.
Det har blitt lagt fokus på et brukervennlig grensesnitt for sjåføren, og systemet skal være
lett å oppdatere og lett å integrere i de eksisterende systemene til lastebiler. 

I markedsanalysen i denne rapporten så er det tenkt at dette produktet kan være et nisje produkt. 
Det må selges 4800 enheter for å dekke utgiftene til produktet og produktutviklingen, noe som
er realiserbart med en rimelig pris på produktet.

For å kunne realisere konseptet kreves grundig testing for å finne ut hvor nøyaktig og pålitelig 
disse vibrasjonssensorene kan være. Siden denne rapporten kun er basert på teori, så kan det 
oppstå uforutsette hindere. Utviklingen av produktet vil være preget av mye testing for å få 
systemet til å gjenkjenne løse hjulmuttere.

Produktet er ment til å komme ferdiginstallert i nye lastebiler og skal gjøre hverdagen enklere for 
lastebilsjåfører, ved at sjåføren slipper å tenke på om hjulmutterene er løse eller ikke. Denne
rapporten utforsker mulighetene ved å bruke vibrasjonssensorer for å detektere løse hjulmuttere,
men denne teknologien åpner også for andre muligheter for å detektere feil ved kjøretøyet.
\section{Eksisterende løsninger}
Det eksisterer i skrivende stund ingen digitale løsninger på dette problemet. 
Derimot finnes det noen mer manuelle løsninger for å unngå at mutre løsner, 
og løsninger for å sjekke stilling på hjulmutre.
\subsection{NordLock}
NordLock er et selskap som utvikler bolter og mutre som bruker trykk for å forhindre 
at gjengene løsner, og at bolt og mutter glir fra hverandre. NordLock har flere 
produkter, deriblant en underlagsskive \ref{fig:NL-skive}\cite{NL-skive} med mothaker mot 
utskruingsretning\comment{TODO: bedre phrasing her?}, og en hjulbolt 
\ref{fig:NL-hjulbolt}\cite{NL-hjulbolt} som er basert på samme prinsipp som 
underlagsskivene, og som dermed, i flg. NordLock, ikke kan løsne av seg selv.
\newline
\begin{figure}[H]
\subfigure[]{
	\includegraphics[width=0.5 \textwidth]{images/NL-skive.jpg}
	\label{fig:NL-skive}
	}%
\hfill
\subfigure[]{
	\includegraphics[width=0.5 \textwidth]{images/NL-hjulbolt.jpg}
	\label{fig:NL-hjulbolt}
	}%
\caption{NordLock: \protect{\ref{fig:NL-skive}} underlagsskive, \protect{\ref{fig:NL-hjulbolt}} hjulbolt.}
\end{figure}


\subsection{Checkpoint}
\cite{checkpoint1}
\subsubsection{Checkpoint}
\subsubsection{Dustite}
\subsubsection{Checklink}
\subsubsection{Checklock}
\subsubsection{Rollock}
\cite{rolllock}

\subsection{Digital visualisering}
Det eksisterer i skrivende stund ingen softwarebaserte visuelle løsninger dette 
problemet. Bildet under viser hvordan Polyphony Digital løste visualisering av 
dekkslitasje i sitt spill Gran Turismo 6.\cite{dekkslitasje-GT6} 
	\newline
	\begin{figure}[H]
		\centering
		\includegraphics[width=1.00\textwidth]{images/gran-turismo-6-screenshot.jpg}
		\caption{Dekkslitasje visualisert i Gran Turismo 6.}
	\end{figure}

\chapter{Forstudie}

\section{Hardware}

Hardware omfatter systemet fra sensor til nettverksprotokollen. Hvilken hardware
som trengs er da avhengig av hvilke sensorer som er tilgjengelige, utfordringer
i forhold til signaloverføring, hvilke data
man får fra ulike sensorer, hvilken dataprosessering som er nødvendig og
hvilken nettverksprotokoll som brukes. \\

\subsection{Definerende og begrensende faktorer}

En definerenede faktor for hardware er måleteknikk. Ulike måleteknikker studeres
i seksjon (henvisning). Måleteknikken setter også en begrensing på hva som er
mulig å måle og dermed hvilke tjenester systemet kan tilby. \\

I bilindustrien er CAN den rådende nettverksprotokollen (sjekke med ref). Man
kan argumentere for at man på grunnlag av dette burde integrere nye systemer på
det eksisterende nettverket. Uansett hvilket nettverk det skal kommuniseres over
vil det stille krav til hardware. En ny enhet bør derfor kunne kommunisere på samme
nettverk på riktig protokoll. \\

\subsection{Sensorteknologier}

Det er mange fysiske prinsipper som kan måles for å detektere en løs hjulmutter.
Det er derfor viktig å undersøke hvilke sensorteknologier som er tilgjengelig og
hvilke fordeler disse drar med seg. \\

Tre ulike fysiske prinsipper som kan brukes er strekk, trykk og vibrasjon

\section{Vibrasjonsmåling}

Ved måling av vibrasjon trenger man en ADC. En DSP kunne blitt brukt for å kjøre
avanserte algoritmer på input signalene, men det kan også blir gjort mer sentral
(ett nivå over) for gjør oppgradering av SW enklere. Konsekvensen er også
laverer effektforbruk og kostnad. \\

Kommunikasjon skjer ved å sende CAN pakker på et eksisterende nettverk. Pakkene
inneholder målinger tatt periodisk med et gitt samplingsvindu. Man må da se
nærmere på hvor stor dette vinduet bør være samt hvor ofte målingen må tas. Det
er kanskje nødvendig med analog filtrering av signalet (før ADC). \\

Tanken er å ha sensoren på akslingen mellom to hjul. For å skille mellom hvilket
hjul som har løse muttere, så kan det være en løsning å bruke en sensor ved vært
hjul. Da er to mulig algoritmer å måle amplitudeforskjell eller
tidsforsinkelse. Dette burde være mulig å få til med en MCU, men kan være det
blir nødvendig med en DSP for raskere prosessering av signalene. \\

MCUen sender pakkene med en ID slik at man kan skille mellom ulike akslinger.

\section{Strekklapper/Trykklapper}

Tanken er å legge strekklapper inni eller på utsiden av boltene. Strekklappene
må/kan/bør legges i en krets (Wheatstone feks) for å kunne gi ut et målbart
nivå. Denne analoge målingen krever mye lavere sampling end ved vibrasjonsmålinger.
Man kan få problemer med at signalet fra strekklappen må gjennom en slepering.
Man må også ha en sensor per bolt. Dette fører til mer HW. Det er mulig å lage
en krets hvor man benytter seg av tidskonstanten i en RC-krets. Overføring av
signaler fra nav til aksling kan også skje ved induksjon. For å koble til flere
enheter på et CAN nettverk så kan man benytte seg av J1939 som fordelere ID på
nettverket.


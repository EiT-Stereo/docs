\section{Vibrasjoner}
\subsection{Absorvering av vibrasjoner}
I et ideelt tilfelle av vibrasjonsmåling burde mediet som skal transportere vibrasjoner 
enten være helt fritt opplagret eller helt fast opplagret. Problemet på en lastebilaskling 
er at den ligger litt midt imellom. Siden en helt fast aksling vil ført til en ganske humpete 
og ristende kjøretur er lastebiler utstyrt med et fjærende oppheng. Fjærene på en lastbil
 tar opp både krefter pga last, samt akselerasjon og deakselerasjon. Fjærene er godt 
festet til akslingen, men er opplagret i et mykere oppheng i endepunktene, gjerne av gummi. 
Disse opplagringene vil ta opp vibrasjoner, noe som kan gjøre at målingene på akslingen vil bli svekket.

\subsection{Hvilken kilde til vibrasjoner vil være den beste?}
%TODO intro til spørsmål
\begin{itemize}
	\item Vil endring av egenfrekvens grunnet løsnende hjul kunne være noe som er 
målbart med god nøyaktighet?
	\item Vil en mutter skru seg helt ut før resten har
 løsnet, og vil den ha nok masse til at ubalansen i hjulet vil være målbar med 
høg nok sikkerhet? 
	\item Eller er et helt løst hjul det eneste som er garantert til å få 
vibrasjoner som er kraftige nok, samtidig som det er gjentagbart?
\end{itemize}
Uten faktisk målinger vil det være vanskelig å si hvilken kilde av vibrasjoner 
som vil være den beste. Vibrasjonsmåling for å detektere ødelagt lager og 
kast i asklinger er en godt etablert viten, men å bruke vibrasjonsmåling for 
å detektere et løst hjul på en lastebil har aldri blitt gjort før, og det kreves 
derfor nærmere undersøkning før noe kan konkluderes med. 

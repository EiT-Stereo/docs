\chapter{Forstudie}

\section{Hardware}

Hardware omfatter systemet fra sensor til nettverksprotokollen som blir brukt i
bilsystemet. Hvilken hardware
som trengs er da avhengig av hvilke sensorer som er tilgjengelige, hvilke data
man får fra de ulike sensorene, hvilken dataprosessering som er nødvendig og
utfordringer i forhold til signaloverføring og
hvilken nettverksprotokoll som brukes. \\

\subsection{Definerende og begrensende faktorer}

En definerenede faktor for hardware er måleteknikk. Det er da særlig krets
mellom eventuelt microcontroller og sensor som vil variere. Ulike måleteknikker studeres
i seksjon (henvisning). Måleteknikken setter også en begrensing på hva som er
mulig å måle og dermed hvilke tjenester systemet kan tilby. \\

I bilindustrien er CAN den rådende nettverksprotokollen (sjekke med ref). Man
kan argumentere for at man på grunnlag av dette burde integrere nye systemer på
det eksisterende nettverket. Uansett hvilket nettverk det skal kommuniseres over
vil det stille krav til hardware. En ny enhet bør ha riktig nettverksgrensesnitt
og må kunne kommunisere på riktig protokoll. \\

\subsection{Sensorteknologier}

% TODO: Teknologi, utfordringer, løsninger

(Denne er seksjonen er tildels uferdig). \\

Det er mange fysiske prinsipper som kan måles for å detektere en løs hjulmutter.
Det er derfor viktig å undersøke hvilke sensorteknologier som er tilgjengelig og
hvilke fordeler disse drar med seg. \\

Tre ulike fysiske prinsipper som kan brukes er strekk, trykk og vibrasjon. \\

\section{Vibrasjonsmåling}

Vibrasjonsmåling kan bli gjort med teknologier som et akselerometer eller ved
pizoelektriske microfoner. Slike sensorer vil kunne gi et signal som inneholder
ulike frekvenstyper som kan analyseres. For å analysere dette signalet må det
samples med en ADC. Dersom signalet er svakt må det også forsterkes gjennom en
OP AMP krets. For å hente ut informasjon av signalet må en algoritme undersøke
frekvensoppbyggingen av signalet. Dette kan bli gjort i HW med en DSP, men kan
også gjøres i SW ved at man bare sender en signalprøve til controlsystemet. Å
bruke en DSP kompliserer HW og effektforbruk og kost øker. Analyserer man
signalet i SW slipper man disse konsekvensene og det blir lettere å oppdatere
algoritmen dersom det er nødvendig. \\

Kommunikasjon skjer ved å sende CAN pakker på et eksisterende nettverk. Pakkene
inneholder målinger tatt periodisk med et gitt samplingsvindu. Man må da se
nærmere på hvor stor dette vinduet bør være samt hvor ofte målingen må tas. Det
er kanskje nødvendig med analog filtrering av signalet (før ADC). \\

Tanken er å ha sensoren på akslingen mellom to hjul. For å skille mellom hvilket
hjul som har løse muttere, så kan det være en løsning å bruke en sensor ved vært
hjul. Da er to mulig algoritmer å måle amplitudeforskjell eller
tidsforsinkelse. Dette burde være mulig å få til med en MCU, men kan være det
blir nødvendig med en DSP for raskere prosessering av signalene. \\

MCUen sender pakkene med en ID slik at man kan skille mellom ulike akslinger. \\

\section{Strekklapper/Trykklapper}

Tanken er å legge strekklapper inni eller på utsiden av boltene eller
trykklapper mellom bolt og nav. Strekklappene
må/kan/bør legges i en krets (Wheatstone feks) for å kunne gi ut et målbart
spenningsnivå. For å oppnå et målbart spenningsnivå må man mest sannsynlig
forsterke signalet gjennom en In Amp. Denne analoge målingen krever mye lavere sampling end ved
vibrasjonsmålinger da det trengs kortere signalprøver.
Man kan få problemer med at signalet fra strekklappen må gjennom en slepering.
Dette kan by på utfordringer med tanke på implementasjon og støy.
Man må også ha en sensor per bolt. Dette fører til mer HW. En måte å redusere
mengden HW på kan være å MUX'e signalene fra hver sensor til MCU. Det er mulig å lage
en krets hvor man benytter seg av tidskonstanten i en RC-krets (dette må du
undersøke litt). Overføring av
signaler fra nav til aksling kan også skje ved induksjon (dette må du undersøke
litt mer). For å koble til flere
enheter på et CAN nettverk så kan man benytte seg av J1939 som fordelere ID på
nettverket (dette må du undersøke litt mer). \\

Rapporten tar for seg automatisk identifikasjon og varsling av løse hjulmuttere, samt hvordan og hvorfor forskjellige
valg har blitt tatt i prosessen av å komme fram til det vi mener er den beste løsningen. Prosjektet er løst av seks studenter ved NTNU
som i samarbeid med Kongsberg Automotive har et håp om å redde liv på alle veier i verden der lastebiler ferdes. 

Et komplett system som i teorien vil identifisere og varsle løse hjulmuttere automatisk har blitt ukviklet, men det blir ikke
gått nærmere inn på testing av systemene. Ressurser og tilgang på utstyr eller lastebiler for å utføre 
nødvendige målinger og tester for å kontrollere at teorien fungerer i praksis har ikke vært tilgjengelig. Ettersom dette er en skoleoppgave med begrenset
kapital innad i teamet har det heller ikke vært mulighet til å kjøpe inn utstyr for å teste prototyper eller lignende. Det har heller ikke 
vært mulig å oppdrive informasjon om hvordan man kan koble seg inn på lastebilens datasystemer.

Alle seks studenter i prosjektet tar mastergradsutdanning på NTNU. Gjennom tidligere prosjekter og skolegang har de derfor
mulighet til å løse tekniske problemer som vil kunne oppstå innenfor alle områder i systemet vi utvikler. De har dog ingen med
økonomisk bakgrunn i teamet, men har likevel gjort et grovt estimat på hvordan markedet vil se ut og hva hardwaren i produktet
vil koste.

I sammarbeid med KA vil systemet selges til deres allerede eksisterende kunder/kontakter. Denne rapporten
inneholder dog ingen utfyllende forretningsplan, men er en drøfting av løsningene og mulighetene som anses som
potensielt profitable, samt en spesifikasjon på "vibrasjonsløsningen''. Denne har blitt valgt ettersom den blir vurdert
til å være den mest innovative og konkurransedyktige løsningen

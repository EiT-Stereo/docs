
Denne rapporten tar for seg automatisk identifikasjon og varsling av løse hjulmuttere, samt døfter hvordan og hvorfor forskjellige
valg har blitt tatt i prosessen av å komme fram til det vi mener er den beste løsningen. Prosjektet er løst av seks studenter ved NTNU
som i samarbeid med Kongsberg Automotive har et håp om å redde liv på alle veier i verden der lastebiler ferdes. 

Vi vil utvikle et komplett system som i teorien vil identifisere og varsle løse hjulmuttere automatisk, men vil i denne rapporten ikke 
gå nærmere innn på testing av systemene. Vi har verken ressurser eller tilgang på utstyr eller lastebiler for å utføre 
nødvendige målinger og tester for å kontrollere at teorien fungerer i praksis. Ettersom dette er en skoleoppgave med begrenset
kapital innad i teamet vil vi heller ikke ha mulighet til å kjøpe inn utstyr for å teste prototyper eller lignende. Vi har heller ikke 
tilgang på informasjon om hvordan vi kan koble oss inn på lastebilens datasystemer, og antar derfor at dette vil være opp til hver
enkelt kunde.

Alle seks studenter i prosjektet tar mastergradsutdanning på NTNU. Gjennom tidligere prosjekter og skolegang har vi derfor
mulighet til å løse tekniske problemer som vil kunne oppstå innenfor alle områder i systemet vi utvikler. Vi har dog ingen med
økonomisk bakgrunn i teamet, men har likevel gjort et grovt estimat på hvordan markedet vil se ut og hva hardwaren i produktet
vil koste.

I sammarbeid med KA vil vi først selge systemet gjennom deres allerede eksisterende kunder/kontakter. Denne rapporten
inneholder dog ingen utfyllende forretningsplan, men er en drøfting av løsningene og mulighetene vi ser på
som potensielt profitable, samt en spesifikasjon på ``vibrasjonsløsningen'' vi har sett oss ut som mest innovativ og konkurransedyktig.
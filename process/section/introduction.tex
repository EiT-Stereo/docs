\chapter{Introduksjon}
\section{Gruppen} 
\subsection{Emil Bjørlykhaug}
Går undervannsteknologi, 2-årig master. Har tidligere gått Automasjon ved Hials.
Som folk flest hadde jeg hørt en del snakk om EiT på forehand, noe positivt, noe negativt. 
Jeg hadde ikke noen store forventninger ang. selve prosjektet vi skulle gjennomføre,
 men hadde en del forventninger til det å jobbe i gruppe, og å lære hvordan gruppedynamikk 
fungerer sett fra både et teoretisk og praktisk perspektiv. Jeg har tidligere selv jobbet 
en del i grupper, men oftest har medlemmene av desse gruppene vært folk jeg har kjent 
fra før. I et tilfelle når jeg studerte på Hials hadde foreleser ansvaret med å dele opp gruppe, 
så jeg havnet på en gruppe med bare ukjente personer. Jeg følte dette er en god måte å 
utfordre folk sosialt, og følte jeg i løpet av semesteret ble utrolig godt kjent med de jeg 
måtte jobbe sammen med.
Jeg er ikke av typen som roper høgest i diskusjoner, og kan ofte bli litt passiv i 
gruppesammenhenger om jeg ikke føler jeg har noe å stille med, men håper 
EiT kan gjøre meg til et bedre gruppemedlem. Jeg er positivt innstilt til faget 
og håper på høgt læringsutbytte.
>>>>>>> master

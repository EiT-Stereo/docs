\chapter{Innledning}

For alle studieprogram på masternivå ved NTNU er Eksperter i Team (EiT) et obligatorisk emne. Emnet går ut på gjøre 
studenten til et bedre gruppemedlem, som definert i læringsmålet til emnet: ``I Eksperter i team lærer studenten å 
kommunisere og samarbeide slik at hun eller han kan
bidra til helhetlige løsninger, trivsel og læring i tverrfaglig prosjektarbeid senere i yrkeslivet.'' \cite{laeringsmaal}

Emnet er delt opp i ulike landsbyer avhengig av hva studenten har lyst å ha som tema i prosjektet. En landsby består 
av ca. 5 grupper på 5-6 studenter med ulike faglig bakgrunn. Temaet for vår landsby er ``IT-styring av moderne 
biler'', og setter en ramme for prosjektarbeidet. \cite{tdt4856} For å oppfylle målet med å gjøre studentene til bedre gruppemedlemmer 
inneholder faget også en prosessdel. Prosessdelen tar for seg det psykologiske aspektet, og teorier rundt hva som får 
et gruppearbeid til å fungere bra.

I løpet i EiT har gruppen støtt på sitasjoner og konflikter innad, og på slutten av 
hver landsbydag satte gruppen av en time for å reflektere over situasjoner som oppstod den dagen. Under refleksjonene 
prøvde vi å komme til bunns i hva som var årsaken til situasjonene for så å iverksette tiltak for å unngå 
at lignende situasjoner skulle oppstå igjen. Vi har i denne rapporten valgt å gjengi de fire sitasjonene, med tilhørende 
refleksjoner og aksjoner, som vi følte formet gruppen mest.

\chapter{Gruppen} 
Gruppen består av seks fjerdeårs studenter ved Norges Teknisk-
Naturvitenskapelige\\Universitet, NTNU, våren 2015, som alle tar faget ``TDT4856 
- Eksperter i Team: IT-styring av moderne biler (med fokus på lastebil)''.	 

\section{Forventninger}
Før semesteret kom i særlig gang, skrev hvert gruppemedlem sine tanker og 
forventninger til faget, samarbeidet og læringsutbyttet. Hvordan disse
forventningene ble møtt kan det leses om i kapittel \ref{individuellerefleksjoner}.

\subsection*{Erlend Aksnes} Flere venner og bekjente hadde fortalt om erfaringer og historier fra EiT som gav meg et dårlig inntrykk av faget som utgangspunkt. 
Jeg har mye erfaring fra gruppearbeid i diverse fag gjennom studietiden.
 På grunn av opplevelser med dårlig ansvarsfordeling i noen fag har jeg blitt mindre interessert i gruppearbeid. 
Til tross for dette ble jeg glad da jeg fikk vite hvilken landsby jeg hadde blitt tildelt og bestemte meg for å gå inn med en positiv innstilling.
Jeg er spent på arbeidet på tvers av fagområder og håper på å kunne lære mer av denne typen gruppearbeid, som er ny for meg.  
Jeg håper at folk er motiverte og prøver å gjøre sitt beste til tross for eventuelle negative tanker om faget. 
I tillegg håper jeg at vi får en god tone i gruppen, da jeg tror dette i stor grad avgjør hvilket resultat vi leverer. 
Som person tror jeg at jeg er lett å samarbeide med, tar ansvar for deler av oppgaven og passer på å gjøre min andel av arbeidet. 

\subsection*{Emil Bjørlykhaug}
Som folk flest hadde jeg hørt en del snakk om EiT på forhånd, noe positivt, noe negativt. 
Jeg fikk en positiv overraskelse når jeg så hvilken landsby jeg havnet i, og gikk derfor 
inn med en positiv innstilling til faget. Jeg har tidligere jobbet en del i grupper, og har 
hatt grei nok erfaringer med det siden jeg ofte har kjent de jeg har vært på gruppe med, 
noe som har ført til lav terskel for å si i fra om noe ikke er som det burde være. 
Jeg tror derimot det skal bli litt spennende å havne på en gruppe med bare ukjente mennesker, 
og hvordan gruppen vil forme seg. Jeg håper det er utadvendte og ansvarsfulle mennesker jeg 
havner på gruppe med slik at det blir god stemning i gruppa, og et bra resultat på prosjektet.

\subsection*{David Hovind} Jeg har ikke hørt så mye om EiT fra før så jeg vet ikke helt hva det går ut på. 
Jeg har derfor et åpent sinn til faget og ser på det som en positiv utfordring. Jeg valgte IT-styring av moderne
biler fordi jeg er veldig interessert i IT-styring og selvstyrte biler, og jeg håper at med min kompetanse innenfor 
kunstig intelligens kan jeg bidra med mye i denne landsbyen. Jeg har erfaring med gruppearbeid fra proskekter 
innen programvareutvikling på NTNU, men jeg har aldri vært med på gruppeprosjekt med forskjellige teknologibakgrunner.
Derfor så håper jeg at det blir en god opplevelse hvor jeg kan lære å jobbe sammen med mennesker som jeg kanskje ikke
har så mye til felles med.

Som regel så foretrekker jeg ikke gruppearbeid fordi man blir avhengig av alle andre i gruppen og de blir avhengige av deg. 
Av den grunn så liker jeg best å jobbe selvstendig, siden det finnes mange forskjellige typer mennesker og man vet aldri 
hvilke typer mennesker man havner på gruppe med. Jeg forventer å bli bedre til å kjenne meg selv og få litt innsikt i 
hvordan det er å reflektere over gruppearbeidet.

\subsection*{Kristoffer Løvall}
Jeg fikk høre litt forskjellig om Eksperter i Team før dette semesteret, men har 
prøvd å holde de negative holdningene til faget ute. Jeg starter derfor semesteret med en positiv innstilling, spesielt siden landsbyens tema 
virker veldig interessant og spennende. Jeg forventer at det blir lærerikt å 
jobbe på tvers av fagfelt og interesser, og at tverrfagligheten bidrar til å 
designe et godt produkt. På grunn av tidligere gruppearbeid i forhåndsbestemte 
grupper med ukjente personer har jeg ikke store forventninger til det sosiale, 
noe som på sitt vis kan være positivt ettersom det da ikke er mye som skal til 
for at dette går over forventning. Jeg har også jobbet relativt mye i grupper 
gjennom porsjekter både på NTNU og Høgskolen i Bergen, der jeg ofte har endt opp 
med å ta ansvar og fungere som en gruppeleder. Jeg vet også at jeg kan bli noe 
kontrollerende når jeg blir veldig engasjert og brenner for noe.  Målet mitt for 
Eksperter i Team er å ``prøve noe nytt'' og være litt mer tilbakeholden på denne 
fronten, prøve å jobbe på linje med alle andre uten å ``ta over'' for noen og 
kontrollere alle detaljer. Jeg vil bli enklere å jobbe i team med og lære meg 
selv å godta andres løsninger og akseptere disse som de beste løsningene, selv om 
min opprinnelige tanke var noe helt annet. Når det gjelder det faglige håper jeg 
alle har høye forventninger både når det gjelder engasjement for å ende opp med 
gode løsninger, samt det å få en god karakter i faget.

\subsection*{Christoffer Ramstad-Evensen}
Fra tidlig i utdanningen ved NTNU har jeg hørt studenter snakke om EiT. Det var tydelig at dette var noe alle hadde 
meninger om, og meningene var forskjellige. Mye av det som ble sagt rettet seg mot at det ikke var særlig morsomt 
å måtte bruke en dag i uka på å reflektere rundt følelser sammen med en gruppe mennesker du ikke kjente. Det 
kunne også virke som at flere studenter reiste utenlands rett å slett for å unngå EiT. Alle meninger var dog ikke 
negative og rettet seg mer i mot et utbytte av faget som var verdifullt med tanke på jobbsammenheng. Dette kunne 
jeg være enig i. Jeg så for meg at det å jobbe sammen i en større gruppe og holde fokus på selve samarbeidet var 
noe jeg kunne vokse på, som person. Det var absolutt en erfaring jeg tenkte jeg ikke kunne få nok av, og som jeg 
anså som verdifull å ta med meg i arbeidslivet. Av erfaring så er det ikke lurt av meg selv å angripe et fag med 
negativ innstilling, så jeg begynte å fortelle meg selv at EiT kom til å bli gøy og gikk faget i møte med en positiv 
innstilling.

\subsection*{Nina Margrethe Smørsgård}
Jeg har hørt mye forskjellig om Eksperter i Team fra eldre studenter. Alt fra 
at det er super kjedelig og mye arbeid fordi mange velger landsby utenfor sitt 
studiefelt for å slippe å gjøre en innsats, til at det er inspirerende og gøy 
å jobbe med folk på tvers av studieretninger. Jeg gikk inn i EiT med en positiv 
holdning, fordi jeg fikk en landsby med et tema jeg synes var veldig 
interessant, så de negative tingene jeg hadde hørt la jeg litt til siden. Jeg 
var allikevel noe redd for at vi skulle få mange med ikke-teknisk bakgrunn selv 
om det var en veldig teknisk tittel på landsbyen.

Jeg har jobbet en god del i grupper på NTNU; både i prosjektfag og andre fag.
Jeg har også hatt sommerjobber som programvareutvikler i flere større norske IT-
bedrifter, hvor vi har jobbet sammen i team på 6-8 studenter hver sommer. 
Derfor tror jeg ikke jeg kommer til å lære så veldig mye nytt om generelt 
gruppearbeid av EiT, men jeg tror heller at jeg får muligheten til å prøve å 
takle arbeid i en gruppe hvor det er folk med andre kompetansefelt enn mitt eget.

Mine forventninger til EiT i vår er å lære mye om mekanikken i en bil, og at 
vi får utvikle et kult konsept og også lage en forhåpentligvis fungerende prototyp.

\section{Gruppedynamikk}
Mennesker har alltid jobbet i grupper, helt fra tidenes morgen. Man jaktet 
sammen, jobbet i små samfunn sammen, og skapte familier der familiemedlemmene 
er gruppemedlemmene og familien er gruppen. Ved å samarbeide i grupper, blir 
livene til de individuelle drastisk bedre. Dette er direkte relatert til 
gruppeeffektivitet, der effektiviteten stiger i takt med graden av 
samarbeid. Effektiviteten er også påvirket av gruppestrukturen. Dette er et 
punkt vi har jobbet godt med underveis, ved å sette konkrete og kortisktige 
mål og som har ført til at strukturen i gruppen har blitt styrket etterhvert 
som disse ble nådd.

I en gruppe har personene forskjellige roller, men 
forskjeller i rollene og oppfatningen av dem kan føre til konflikt. Rollene 
definerer den formelle strukturen av gruppen, men disse trenger ikke stemme 
overens med de faktiske rollene. Selv om en leder er definert som en leder, 
kan det være andre personer i gruppen som tar lederrollen i praksis. 
Vi har i dette prosjektarbeidet forsøkt å holde de formelle rollene åpne, 
altså å kjøre samarbeidet uten en formell leder. Dette har vist seg å 
fungere godt på noen punkter, men det har også vist seg at enkelte 
beslutninger kan bli vanskelige å ta. 

Vi hadde et håp om å ende opp som det 
David og Frank Johnson (2013) kaller en ``High perfomace group'', altså en 
gruppe som møter alle kriteriene til en effektiv gruppe, og i tillegg til 
dette produserer arbeid langt over forventninger. Vi føler dog vi ikke har 
klart dette til det fulle, blant annet på grunn av den lederløse
strukturen, men vi er alle enige om at vi har havnet i gruppetypen 
``Effective group''. Dette er en gruppe der medlemmene jobber sammen for å 
nå et felles mål, og oppfatter det å nå egne mål som mulig kun om de andre 
gruppemedlemmene også når sine mål. \cite{Artikkel2}

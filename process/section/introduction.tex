\chapter{Introduksjon}
\section{Gruppen} 
\subsection{Emil Bjørlykhaug}
Går undervannsteknologi, 2-årig master. Har tidligere gått Automasjon ved Hials.
Som folk flest hadde jeg hørt en del snakk om EiT på forehand, noe positivt, noe negativt. 
Jeg hadde ikke noen store forventninger ang. selve prosjektet vi skulle gjennomføre,
 men hadde en del forventninger til det å jobbe i gruppe, og å lære hvordan gruppedynamikk 
fungerer sett fra både et teoretisk og praktisk perspektiv. Jeg har tidligere selv jobbet 
en del i grupper, men oftest har medlemmene av desse gruppene vært folk jeg har kjent 
fra før. I et tilfelle når jeg studerte på Hials hadde foreleser ansvaret med å dele opp gruppe, 
så jeg havnet på en gruppe med bare ukjente personer. Jeg følte dette er en god måte å 
utfordre folk sosialt, og følte jeg i løpet av semesteret ble utrolig godt kjent med de jeg 
måtte jobbe sammen med.
Jeg er ikke av typen som roper høgest i diskusjoner, og kan ofte bli litt passiv i 
gruppesammenhenger om jeg ikke føler jeg har noe å stille med, men håper 
EiT kan gjøre meg til et bedre gruppemedlem. Jeg er positivt innstilt til faget 
og håper på høgt læringsutbytte.

\subsection{David Hovind} Jeg har ikke hørt så mye om EiT fra før så jeg vet ikke helt hva det går ut på. 
Jeg har derfor et åpent sinn til faget og ser på det som en positiv utfordring. 
Som regel så foretrekker jeg ikke gruppearbeid fordi man blir avhengig av alle andre i gruppen og de blir avhengige av deg. 
Av den grunn så liker jeg best å jobbe selvstendig, siden det finnes mange forskjellige typer mennesker og man vet aldri 
hvilke typer mennesker man havner på gruppe med. Jeg forventer å bli bedre til å kjenne meg selv og få litt innsikt i 
hvordan det er å reflektere over gruppearbeidet.

\subsection{Kristoffer Løvall}
Jeg fikk høre litt forskjellig om Eksperter i Team før dette semesteret, men har 
prøvd å holde de negative holdningene til faget ute. Jeg starter semesteret 
derfor semesteret med en positiv innstilling, spesielt siden landsbyens tema 
virker veldig interessant og spennende. Jeg forventer at det blir lærerikt å 
jobbe på tvers av fagfelt og interesser, og at tverrfagligheten bidrar til å 
designe et godt produkt. På grunn av tidligere gruppearbeid i forhåndsbestemte 
grupper med ukjente personer har jeg ikke store forventninger til det sosiale, 
noe som på sitt vis kan være positivt ettersom det da ikke er mye som skal til 
for at dette går over forventning. Jeg har også jobbet relativt mye i grupper 
gjennom porsjekter både på NTNU og Høgskolen i Bergen, der jeg ofte har endt opp 
med å ta ansvar og fungere som en gruppeleder. Jeg vet også at jeg kan bli noe 
kontrollerede når jeg blir veldig engasjert og brenner for noe.  Målet mitt for 
Eksperter i Team er å "prøve noe nytt" og være litt mer tilbakeholden på denne 
fronten, prøve å jobbe på linje med alle andre uten å "ta over" for noen og 
kontrollere alle detaljer. Jeg vil bli enklere å jobbe i team med og lære meg 
selv å godta andres løsninger og akseptere disse som de beste løsningene, selv om 
min opprinnelige tanke var noe helt annet. Når det gjelder det faglige håper jeg 
alle har høye forventninger både når det gjelder engasjement for å ende opp med 
gode løsninger, samt det å få en god karakter i faget.

\subsection*{Nina Margrethe Smørsgård}
Jeg har hørt mye forskjellig om Eksperter i Team fra eldre studenter. Alt fra 
at det er super kjedelig og mye arbeid fordi mange velger landsby utenfor sitt 
studiefelt for å slippe å gjøre en innsats, til at det er inspirerende og gøy 
å jobbe med folk på tvers av studieretninger. Jeg gikk inn i EiT med en positiv 
holdning, fordi jeg fikk en landsby med et tema jeg synes var veldig 
interessant, så de negative tingene jeg hadde hørt la jeg litt til siden. Jeg 
var allikevel noe redd for at vi skulle få mange med ikke-teknisk bakgrunn selv 
om det var en veldig teknisk tittel på landsbyen.

Jeg har jobbet en god del i grupper på NTNU; både i prosjektfag og andre fag.
Jeg har også hatt sommerjobber som programvareutvikler i flere større norske IT-
bedrifter, hvor vi har jobbet sammen i team på 6-8 studenter hver sommer. 
Derfor tror jeg ikke jeg kommer til å lære så veldig mye nytt om generelt 
gruppearbeid av EiT, men jeg tror heller at jeg får muligheten til å prøve å 
takle arbeid i en gruppe hvor det er folk med andre kompetansefelt enn mitt eget.

Mine forventninger til EiT i vår er å lære mye om mekanikken i en bil, og at 
vi får utvikle et kult konsept og også lage en forhåpentligvis fungerende prototyp.
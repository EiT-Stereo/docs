\chapter{Tema 1: Lunsj}
Det første temaet vi presenterer omhandler lunsjpauser og hvordan lunsjpausene påvirker resten av arbeidet. I starten av EiT kjente ingen av gruppemedlemmene hverandre, det ble derfor naturlig å tilbringe lunsjpausene sammen for å bli bedre kjent med hverandre uten å prate om oppgaven. I starten av prosjektet ble vi enige om å ha felles lunsj, hvorvidt denne avtalen var skriftlig, muntlig, eller bare et åpent tilbud var det forskjellig oppfatning av.

\section{Situasjonsbeskrivelse}
Første gangen dette temaet ble tatt opp som en situasjon vi burde reflektere over var det Nina som ikke deltok på felles lunsj. På dette tidspunktet var det flere av gruppemedlemmene som ikke husket eller var klar over avtalen om felles lunsj. 

Den andre gangen vi valgte å reflektere over dette emnet tok Erlend litt tidlig lunsj med noen kamerater etter at gruppen sa at dette var greit. Siden David jobbet med Erlend denne dagen tok også han lunsj litt tidlig på samme tidspunkt.

Den tredje gangen emnet ble tatt opp var det på grunn av fraværende gruppemedlemmer, Erlend og Nina. 




\Section{Refleksjoner og aksjoner}

Aksjon:
I revidert samarbeidskontrakt skrev vi følgende: ``Hver morgen bestemmes et tidspunkt og tidsrom for felles pause.''

\Section{Evaluering av aksjon} 

her

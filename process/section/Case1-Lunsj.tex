\chapter{Tema 1: Lunsj}
Det første temaet vi presenterer omhandler lunsjpauser og hvordan en uenighet angående en lunsjpause og hvorvidt det var viktig å ha pausen samlet eller ikke påvirker resten av arbeidet. I starten av EiT kjente ingen av gruppemedlemmene hverandre, det ble derfor enighet om å tilbringe lunsjpausene sammen for å bli bedre kjent med hverandre uten å prate om oppgaven. Vi ble derfor enige om å skrive inn dette i arbeidsavtalen, men hvorvidt denne avtalen var skriftlig, muntlig, eller bare var et åpent tilbud var det forskjellig oppfatning av.

\section{Situasjonsbeskrivelse}
Temaet ble tatt opp som en situasjon vi burde reflektere over på landsbydag ????????? da Nina ikke deltok på felles lunsj. På dette tidspunktet var det flere av gruppemedlemmene som ikke husket eller var klar overAt dette var en skriftlig avtale. Ved enkelttileffer vil ikke dette ha noe særlig betydning, men kan få konsekvenser dersom et eller flere av gruppemedlemmer ikke deltar på fellesj lunsj ofte. Konsekvenser kan da være at man ikke blir så godt kjent med hverandre og at samholdet i gruppen svekkes. 

At Nina ikke deltok på lunsjen denne dagen ble oppfattet forskjellig av gruppemedlemmene. Noen mente hun var irritert eller hadde en dårlig dag og derfor trakk seg unna for ikke å påvirke oss andre, mens andre var redd hun mislikte gruppen og derfor ikke ville tilbringe fritiden sin med disse gruppemedlemmene. 


%Senere i prosjektetarbeidet hendte det at forskjellige gruppemedlemmer var syke eller bortreist, som igjen førte til variert størrelse på gruppen noen dager.  Det ble nevnt av flere gruppemedlemmer at det var lettere å holde en samtale med en gruppe som ikke er så stor. Grupper med seks medlemmer er så store at det kan føre til indre grupperinger og at samtalene deler seg opp.

\section{Refleksjoner og aksjoner}
Vi vil her gå igjennom hvilke tanker gruppemedlemmene hadde om temaet og hvilke aksjoner vi tok i bruk for å forbedre situasjonen.

\subsection*{Refleksjoner}
Det ble tidlig klart at vi hadde forskjellig oppfatning av hvor obligatorisk den felle lunsjpause var. Vi lærte at beskjeder og avtaler burde være tydelige, slik at alle får de med seg og har en felles forståelse av avtalen. 

%Vi ble enige om at det vil være lurt å ha felles lunsj stående som et åpent tilbud, med mulighet for å gjøre andre ting i lunsjpausen, men var  

\begin{quote}``
Jeg var ikke klar over at vi hadde en avtale om felles lunsj, men hadde uansett tenkt å spise lunsjen ved gruppebordet.
''\end{quote} \rightline{{\rm --- Erlend}}

\begin{quote}``
Man kan vel egentlig ikke tvinge noen til å gjøre noe de ikke har lyst til. Det er bedre å ta tid for seg selv enn å sitte å lage dårlig stemning, men for å prøve å skape et bedre sosialt samhold i gruppen er det flott om vi kan få til en felles rutine på pausene.
''\end{quote} \rightline{{\rm --- Kristoffer}}


%%%%%% alt under her må endres 

Selv om vi satt dette som et åpent tilbud, var det stor enighet i gruppen om at det ville være gunstig for sammarbeidet å prøve opprettholde en felles rutine på lunsjen. Likevel ble ikke dette opprettholdt det kommende landsbydagene. På landsbydag 7 valgte Erlend å møte en venn i lunsjpausen. Han tok derfor lunsj kl 11, og ettersom David jobbet med Erlend denne dagen, valgte han å ta lunsj alene til samme tid. De fleste i gruppen mente det var for tidlig å ta pausen såpass tidlig, og bestemte seg for at alle skulle kjøre eget løp denne dagen, og dermed velge tidspunkt for pause selv. Dette viste seg å ikke fungere spesielt godt for noen. 



\begin{quote}``
Synes det er helt greit at Erlend velger å møte en kompis, jeg ville nok gjort det samme selv, men jeg synes også det var kjipt at vi andre ikke hadde lunsjen sammen. Siden det løsner godt opp på stemningen etter Peppes-kvelden forrige uke, virker det som folk har blitt bedre kjent og at samarbeidet fungerer bedre enn det tidligere har gjort. Jeg tenker det kunne vært ålreit å spist lunsj sammen med den resterende gjengen ettersom alle kommer såpass godt overens. Jeg hadde ikke lunsjpause bortsett fra rundt 10 min der jeg spiste, og ble derfor relativt tung i hodet utover dagen, men dette er jo såklart mitt eget ansvar.
''\end{quote} \rightline{{\rm --- Kristoffer}}

Etter dette forstod gruppen at en felles pause var nokså viktig, ikke bare for det sosiale, men også for strukturen i arbeidet. Vi merket at vi ble flinkere og flinkere på å planlegge denne ettersom tiden gikk, og opprettet en ordning der vi bestemte oss for når og hvor vi skulle ha lunsjpause under morgenmøtet. Dette har vist seg å fungere godt, og vi har som regel vært samlet etter dette, med unntak av sykdomsfravær og lignende. Gjennom det videre sammarbeidet i faget, har vi merket at det sosiale samholdet fikk en bratt opptur etter at denne regelen ble innført. Vi har også i noen tilfeller lagt merke til hvordan gruppestørrelsen påvirker både hvordan vi jobbet sammen, og hvordan vi sosialiserte sammen. 

\begin{quote}``
Kan begynne med å nevne at under prosessartikkelpresenteringen viste det seg at teorien sier at grupper på 6 og oppover begynner å bli i største laget da antall relasjoner utvider seg eksponensielt. Vi liker både Erlend og Nina godt, så det at det i dag var lettere å samle alle og at alle skravlet like mye under lunsjen, har nok ikke noe med de å gjøre. Rent psykisk er det lettere å inkludere og selv vere inkludert når gruppa blir mindre.
''\end{quote} \rightline{{\rm --- Emil}}

\begin{quote}``
Opplevde det som positivt at vi klarte holde alle som var til stede sammen under lunsjen, spesielt fordi vi var såpass få tilstede på EiT i dag. Det var flere ting jeg merket meg med dagens lunsj. Det å komme seg ut av realfagsbygget (vi spiste på Hangaren) gjorde at det føltes mer som et skikkelig avbrekk enn det tidligere har gjort. Det føltes også lettere enn normalt å holde praten i gang, noe jeg tenker kan være fordi vi er en mindre gruppe. Dette med gruppestørrelser og samhold ble også nevnt i gjennomgangen av prossessartiklene fra EiT-kompendiet tidligere på dagen, og jeg merket at selv om både Nina og Erlend såklart ble savnet under lunsjen, er det lettere å forholde seg til 3 andre enn 5 andre under arbeidet.
''\end{quote} \rightline{{\rm --- Kristoffer}}

Kristoffer sier at det var lettere samarbeide og forholde seg til de andre de dagene teamets representanter var få. 
Dette går inn på et begrep som er kjent innen et team, og som er en definerende faktor for et team, nemlig relasjoner. 
En relasjoner er et forhold mellom to personer og å opprettholde gode relasjoner mellom mange personer samtidig
kan være vanskelig. Et team på seks personer er i følge teorien  \cite{Artikkel4} et litt for stort team med tanke på antallet relasjoner
blant dens medlemmer. Det må derfor jobbes bra med å holde gode relasjoner for dersom én relasjon svikter kan det få 
konsekvenser for hele teamet. 

Vi har merket en tydelig sammenheng mellom dette, noe vi mener understreker viktigheten av teambuilding innad i gruppen. Dette er spesielt overførbart til arbeidslivet senere der man på samme måte som i EiT blir plassert i team med personer man i noen tilfeller ikke kjenner på forhånd.

\subsection*{Aksjoner}
Det kan være vanskelig å skille mellom arbeid og pause, vi kom derfor til enighet om å ikke ha lunsj ved arbeidsbordet for å ha et klart skille mellom arbeid og pause. 
Vi ble enige om at vi burde planlegge på starten av dagen et tidpunkt for en felles lunsjpause. Det skaper et godt samhold i gruppa å ha en felles pause, og det virket som at gruppen var enige i dette.  Da vi skulle revidere samarbeidsavtale la vi derfor til følgende regel:  ``Hver morgen bestemmes et tidspunkt og tidsrom for felles pause.'' Dette viste seg å være en veldig god regel som samlet gruppen.

\section{Evaluering av aksjon} 
Etterhvert ble felles lunsj normalen for gruppen, og alle tok del i felles lunsj når de hadde anledning. Grunnen til dette var nok på grunn av morgenmøtene hvor vi planla dagen og når vi skulle ha lunsj hver dag. Dette gjorde det mye lettere å ha en felles lunsj, bedre struktur Et bedre samhold som følge av felles lunsj førte til en lettere stemning resten av dagen og løsnet gruppemedlemmenes smilebånd.
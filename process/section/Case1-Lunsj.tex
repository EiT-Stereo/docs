\chapter{Tema 1: Lunsj}
Det første temaet vi presenterer omhandler lunsjpauser og hvordan lunsjpausene påvirker resten av arbeidet. I starten av EiT kjente ingen av gruppemedlemmene hverandre, det ble derfor naturlig å tilbringe lunsjpausene sammen for å bli bedre kjent med hverandre uten å prate om oppgaven. I starten av prosjektet ble vi enige om å ha felles lunsj, hvorvidt denne avtalen var skriftlig, muntlig, eller bare et åpent tilbud var det forskjellig oppfatning av.

\section{Situasjonsbeskrivelse}
Temaet ble først tatt opp som en situasjon vi burde reflektere over en dag da Nina ikke deltok på felles lunsj. På dette tidspunktet var det flere av gruppemedlemmene som ikke husket eller var klar over avtalen om felles lunsj. Ved enkelttileffer vil ikke dette ha noe særlig betydning men kan få konsekvenser dersom et eller flere av gruppemedlemmer ikke deltar på fellesj lunsj ofte. Konsekvensene for at man ikke har felles lunsj vil resultere i at man ikke blir så godt kjent med hverandre og gjerne ikke får så godt samhold i gruppen. 

Den andre gangen vi valgte å reflektere over dette emnet, var en gang Erlend tok litt tidlig lunsj med noen kamerater etter at gruppen sa at dette var greit. Siden David jobbet med Erlend denne dagen tok også han lunsj litt tidlig på samme tidspunkt. De andre gruppemedlemmene valgte også å spise til litt forskjellige tidspunkter, siden vi allerede var litt oppdelt i forskjellige arbeidsoppgaver. Dette resultere blant annet i at Kristoffer kun tok en 10 minutters pause for å spise lunsjen sin. 

Senere i prosjektetarbeidet hendte det at forskjellige gruppemedlemmer var syke eller bortreist, som igjen førte til variert størrelse på gruppen noen dager.  Det ble nevnt av flere gruppemedlemmer at det var lettere å holde en samtale med en gruppe som ikke er så stor. Grupper med seks medlemmer er så store at det kan føre til indre grupperinger og at samtalene deler seg opp.

\section{Refleksjoner og aksjoner}
Vi vil her gå igjennom hvilke tanker gruppemedlemmene hadde om temaet og hvilke aksjoner vi tok i bruk for å forbedre situasjonen.

\subsection*{Refleksjoner}
Det ble tidlig klart at vi hadde forskjellig oppfatning av hvor obligatorisk den felle lunsjpause var. Vi lærte at beskjeder og avtaler burde være tydelige, slik at alle for det med seg og har en felles forståelse av avtalen. Første gangen vi reflekterte over lunsjpausen ble vi enige om at det vil være lurt å ha felles lunsj stående som et åpent tilbud, med mulighet for å gjøre andre ting i lunsjpausen. 

\begin{quote}``
Jeg var ikke klar over at vi hadde en avtale om felles lunsj, men hadde uansett tenkt å spise lunsjen ved gruppebordet.
''\end{quote} \rightline{{\rm --- Erlend}}

Vi oppdaget hvor viktig det var å ha felles lunsj samtidig den gangen vi delte oss opp og Kristoffer endte opp med kun 10 minutters lunsjpause. 

Det ble nevnt av flere gruppemedlemmene at det var enklere for gruppen å holde en samtale i lunsjen når man var litt færre enn seks personer. 

\begin{quote}``
Opplevde det som positivt at vi klarte holde alle som var til stede sammen under lunsjen, spesielt fordi vi var såpass få tilstede på EiT i dag. Det var flere ting jeg merket meg med dagens lunsj. Det å komme seg ut av realfagsbygget (vi spiste på Hangaren) gjorde at det føltes mer som et skikkelig avbrekk enn det tidligere har gjort. Det føltes også lettere enn normalt å holde praten i gang, noe jeg tenker kan være fordi vi er en mindre gruppe. Dette med gruppestørrelser og samhold ble også nevnt i gjennomgangen av prossessartiklene fra EiT-kompendiet tidligere på dagen, og jeg merket at selv om både Nina og Erlend såklart ble savnet under lunsjen, er det lettere å forholde seg til 3 andre enn 5 andre under arbeidet.
''\end{quote} \rightline{{\rm --- Kristoffer}}


\subsection*{Aksjoner}
Det kan være vanskelig å skille mellom arbeid og pause, vi kom derfor til enighet om å ikke ha lunsj ved arbeidsbordet for å ha et klart skille mellom arbeid og pause. 
Vi ble enige om at vi burde planlegge på starten av dagen et tidpunkt for en felles lunsjpause. Det skaper et godt samhold i gruppa å ha en felles pause, og det virket som at gruppen var enige i dette.  Da vi skulle revidere samarbeidsavtale la vi derfor til følgende regel:  ``Hver morgen bestemmes et tidspunkt og tidsrom for felles pause.'' Dette viste seg å være en veldig bra regel som samlet gruppen.

\section{Evaluering av aksjon} 
Etterhvert ble felles lunsj normalen for gruppen, og alle tok del i felles lunsj når de hadde anledning. Grunnen til dette var nok på grunn av morgenmøtene hvor vi planla dagen og når vi skulle ha lunsj hver dag. Dette gjorde det mye lettere å ha en felles lunsj, bedre struktur Et bedre samhold som følge av felles lunsj førte til en lettere stemning resten av dagen og løsnet gruppemedlemmenes smilebånd.
\chapter{Tema 1: Lunsj}

%% Skriv om?

Det første temaet vi presenterer omhandler en avtale om felles lunsjpause og hvordan dårlige avtalerutiner og svekket kommunikasjon førte til en situasjon. I starten av EiT kjente ingen av gruppemedlemmene hverandre, det ble derfor enighet om å tilbringe lunsjpausene sammen for å bli bedre kjent med hverandre uten å prate om oppgaven. I alle grupper finnes det normer, der normene bestemmer hvordan gruppemendlemmene skal oppføre seg, eller ikke oppføre seg. \cite{Artikkel2} I en situasjoner der personer ikke kjenner hverandre finnes det utallige normer om hvordan man skal oppføre seg ovenfor hverandre. Det kan anses som normalt å være litt tilbaketrukken for så å vise mer og mer av sin egen personlighet etterhvert som man ser hvordan de andre personene i gruppen reagerer på utspill og meninger. I denne situasjonen viste det seg at disse normene førte til at meninger og følelser ikke ble ytret, men kom fram under gjeldende dags refleksjoner

\section{Situasjonsbeskrivelse}

Under en samtale i oppstartsfasen av faget der vi planla samarbeidet og hvordan vi ville utføre gruppearbeidet, ble det gjort en avtale om at vi burde holde sammen i lunsjpausen.

%Prosessteori om hvordan samholdet i gruppen påvirker arbeidet

Temaet ble tatt opp som en situasjon vi burde reflektere over på landsbydag 4 da Nina ikke deltok på felles lunsj. På dette tidspunktet var det flere av gruppemedlemmene som ikke husket eller var klar over at dette var en avtale. Ved enkelttilfeller vil ikke dette ha noe særlig betydning, men kan få konsekvenser dersom et eller flere av gruppemedlemmer ikke deltar på fellesj lunsj gjentatte ganger. Konsekvenser kan da være at man ikke blir så godt kjent med hverandre og at samholdet i gruppen svekkes. 

\section{Refleksjoner og aksjoner}

Da denne situasjonen ble tatt opp var det helt klart at det kom som en overraskelse på Nina. Hun følte selv det var dårlig gjort av henne, og beklaget ovenfor oss andre at hun hadde stukket av uten forvarsel. Vi lærte at beskjeder og avtaler burde være tydelige, slik at alle får de med seg og har en felles forståelse av avtalen. Det var også viktig for oss andre å ytre at dette helt klart ikke var Ninas feil, og at kommunikasjonssvikt var problemet, ikke henne. 

\begin{quote}``
Jeg var ikke klar over at vi hadde en avtale om felles lunsj, men hadde uansett tenkt å spise lunsjen ved gruppebordet.
''\end{quote} \rightline{{\rm --- Erlend}}

Det var helt tydelig at flere hadde samme oppfatning som Erlend og det ble derfor ytret et forslag om å innføre en avtale som også skulle skrives inn i en revidert samarbeidsavtale. At Nina ikke deltok på lunsjen denne dagen ble oppfattet forskjellig av gruppemedlemmene. Noen mente hun var irritert eller hadde en dårlig dag og derfor trakk seg unna for ikke å påvirke oss andre, mens andre var redd hun mislikte gruppen og derfor ikke ville tilbringe fritiden sin med disse gruppemedlemmene. På grunn av disse forskjellige tolkningene av situasjonen ble det diskutert i hvilken grad avtalen skulle bli håndhevet i særsituasjoner. 

\begin{quote}``
Man kan vel egentlig ikke tvinge noen til å gjøre noe de ikke har lyst til. Det er bedre å ta tid for seg selv enn å sitte å lage dårlig stemning, men for å prøve å skape et bedre sosialt samhold i gruppen er det flott om vi kan få til en felles rutine på pausene.
''\end{quote} \rightline{{\rm --- Kristoffer}}

\begin{quote}``
Jeg var ikke klar over at man måtte spørre om lov får å spise lunch med andre. Det er første gang jeg har vært i en gruppearbeidssituasjon på skolen hvor dette har vært et tema, så jeg tenkte ikke over at det var noe folk kom til å reagere på. Det var i alle fall greit at det ble tatt opp, så jeg ble oppmerksom på det.
''\end{quote} \rightline{{\rm --- Nina}}

Vi ble enige om at det skulle skrives inn i samarbeidsavtalen som et punkt, men at det skulle være lav terskel for å gi ``godkjenning'' for at en person er fraværende. Vi forstod at en felles pause var nokså viktig, ikke bare for det sosiale, men også for strukturen i arbeidet. Vi merket at vi ble flinkere og flinkere til å planlegge denne ettersom tiden gikk, og opprettet en ordning der vi bestemte oss for når og hvor vi skulle ha lunsjpause under morgenmøtet. Vi mente også det ville være gunstig å spise lunsj i et annet lokale enn der vi jobbet, for å kunne skille mer mellom arbeid og pause.

% Trenger meeeeeeeeeeer

\section{Evaluering av aksjon} 

Vi ble enige om at vi burde planlegge på starten av dagen et tidpunkt for en felles lunsjpause. Det skaper et godt samhold i gruppa å ha en felles pause.  Da vi skulle revidere samarbeidsavtalen la vi derfor til følgende regel:  ``Hver morgen bestemmes et tidspunkt og tidsrom for felles pause.'' Denne regelen ga rom for tolkning når det gladt hvorvidt det var \emph{krav} til å være med på denne pausen, slik at man kunne spørre om å trekke seg unna om nødvendig. I tillegg kan være vanskelig å skille mellom arbeid og pause, og vi kom derfor til enighet om å konsekvent spise i et annet lokale for å skape et klarere skille. Disse tiltakene har vist seg å fungere godt, og vi har som regel vært samlet etter dette, med unntak av sykdomsfravær og lignende. Gjennom det videre sammarbeidet i faget, har vi merket at det sosiale samholdet fikk en bratt opptur etter at denne regelen ble innført. En dag der Erlend og Nina var fraværende ble det reflektert rundt lunsjpausen igjen.

\begin{quote}``
Opplevde det som positivt at vi klarte holde alle som var til stede sammen under lunsjen, spesielt fordi vi var såpass få tilstede på EiT i dag. Det var flere ting jeg merket meg med dagens lunsj. Det å komme seg ut av Realfagbygget (vi spiste på Hangaren) gjorde at det føltes mer som et skikkelig avbrekk enn det tidligere har gjort. Det føltes også lettere enn normalt å holde praten i gang, noe jeg tenker kan være fordi vi er en mindre gruppe. Dette med gruppestørrelser og samhold ble også nevnt i gjennomgangen av prossessartiklene fra EiT-kompendiet tidligere på dagen, og jeg merket at selv om både Nina og Erlend såklart ble savnet under lunsjen, er det lettere å forholde seg til 3 andre enn 5 andre under arbeidet.
\end{quote} \rightline{{\rm --- Kristoffer}}

Som refleksjonen over viser har vi i noen tilfeller også lagt merke til hvordan gruppestørrelsen påvirker både hvordan vi jobbet sammen, og hvordan vi sosialiserte sammen. Kristoffer sier at det var lettere samarbeide og forholde seg til de andre de dagene gruppens representanter var få. Dette går inn på et begrep som er kjent innen et team, og som er en definerende faktor for et team, nemlig relasjoner. 
En relasjon er en forhold mellom to personer og å opprettholde gode relasjoner mellom mange personer samtidig
kan være vanskelig. Et team på seks personer er i følge teorien  \cite{Artikkel4} et litt for stort team med tanke på antallet relasjoner
blant dens medlemmer. Det må derfor jobbes bra med å holde gode relasjoner for dersom én relasjon svikter kan det få 
konsekvenser for hele teamet. 

\begin{quote}``
Teorien sier at grupper på 6 og oppover begynner å bli i største laget da antall relasjoner utvider seg eksponensielt. Vi liker både Erlend og Nina godt, så det at det i dag var lettere å samle alle og at alle skravlet like mye under lunsjen, har nok ikke noe med de å gjøre. Rent psykisk er det lettere å inkludere og selv vere inkludert når gruppa blir mindre.
''\end{quote} \rightline{{\rm --- Emil}}

Som Emil reflekterer ligger det mye påvirkning i gruppestørrelse, og ettersom dette kan forurense evalueringen av de tidligere nevnte aksjonene for problemene rundt felles lunsjpause, har vi valgt å komplimentere denne evalueringen med refleksjoner og teori rundt dette. Det kan være vanskelig å vite hva som har påvirket gruppearbeidet i størst grad, ettersom flere faktorer spiller inn, men uavhengig av årsak er det en helt klar positiv trend når det gjelder samarbeidsvillighet og sosialt samhold i gruppa etterhvert som ukene gikk.

Med tiden ble det mer og mer naturlig med felles lunsj, der alle gjorde sitt beste for å få til dette. Dette var nok på grunn av morgenmøtene der dagen ble planlagt, og lunsjtiden fastsatt. Dette gjorde det mye lettere å oppretholde avtalen, altså ble det en bedre struktur rundt dette. Et bedre samhold som følge av felles lunsj førte til en lettere stemning resten av dagen og løsnet gruppemedlemmenes smilebånd. Dette mener vi understreker viktigheten av teambuilding, noe som er spesielt overførbart til arbeidslivet senere der man på samme måte som i EiT blir plassert en gruppe med personer man i noen tilfeller ikke kjenner på forhånd. 

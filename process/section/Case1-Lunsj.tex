\chapter{Tema 1: Lunsj}
Det første temaet vi presenterer omhandler lunsjpauser og hvordan lunsjpausene påvirker resten av arbeidet. I starten av EiT kjente ingen av gruppemedlemmene hverandre, det ble derfor naturlig å tilbringe lunsjpausene sammen for å bli bedre kjent med hverandre uten å prate om oppgaven. I starten av prosjektet ble vi enige om å ha felles lunsj, hvorvidt denne avtalen var skriftlig, muntlig, eller bare et åpent tilbud var det forskjellig oppfatning av.

\section{Situasjonsbeskrivelse}
Temaet ble først tatt opp som en situasjon vi burde reflektere over en dag da Nina ikke deltok på felles lunsj. På dette tidspunktet var det flere av gruppemedlemmene som ikke husket eller var klar over avtalen om felles lunsj. Ved enkelttileffer vil ikke dette ha noe særlig betydning men kan få konsekvenser et eller flere gruppemedlemmer ikke deltar på fellesj lunsj ofte, eller hver gang. Konsekvensene for at man ikke har felles lunsj vil resultere i at man ikke blir så godt kjent med hverandre og gjerne ikke får så godt samhold i gruppen. 

Den andre gangen vi valgte å reflektere over dette emnet, var en gang Erlend tok litt tidlig lunsj med noen kamerater etter at gruppen sa at dette var greit. Siden David jobbet med Erlend denne dagen tok også han lunsj litt tidlig på samme tidspunkt. De andre gruppemedlemmene valgte også å spise til litt forskjellige tidspunkter, da vi allerede var litt oppdelt i forskjellige arbeidsoppgaver. Dette resultere da i at Kristoffer kun tok en 10 minutters pause for å spise lunsjen sin. 

Den tredje gangen emnet ble tatt opp var det på grunn av endring i hvor gruppen tok lunsj, samt fraværende gruppemedlemmer, Erlend og Nina. 


\section{Refleksjoner og aksjoner}
Vi vil her gå igjennom hvilke tanker gruppemedlemmene hadde om temaet og hvilke aksjoner vi tok i bruk for å forbedre situasjonen.

\subsection*{Refleksjoner}
Det ble tidlig klart at vi hadde forskjellig oppfatning av hvor obligatorisk felles lunsjpause var. Vi lærte at vi beskjeder og avtaler burde være tydelige, slik at alle for det med seg og har en felles forståelse av avtalen. Under første gang vi reflekterte over lunsjpausen ble vi enige om at det vil være lurt å ha felles lunsj stående som et åpent tilbud, med mulighet for å gjøre andre ting i lunsjpausen. 

\subsection*{Aksjoner}
Det kan være vanskelig å skille mellom arbeid og pause, vi kom derfor til enighet om å ikke ha lunsj ved arbeidsbordet for å ha et klart skille mellom arbeid og pause. 
Ved revidering av samarbeidsavtale la vi til følgende regel:  ``Hver morgen bestemmes et tidspunkt og tidsrom for felles pause.''

\section{Evaluering av aksjon} 

her

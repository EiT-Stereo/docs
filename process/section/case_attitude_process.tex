\chapter{Tema 2: Holdinger til grupperefleksjon}

Under følgende tema vil det diskuteres gruppens og enkeltpersoners holdninger til prosessdelen av EiT og hvordan
disse har påvirket samarbeidet i gruppen. Det vil legges frem hvordan tydligheten av todelingen med prosess og 
prosjekt kom fram i vårt samarbeid og hva som var læringsutbytte fra situasjonen. (Add en liten sånn: Ble det bedre eller verre..)

\section{Situasjonsbeskrivelse}

Temaet ble tatt opp når Nina ble konfrontert med sin gjentagende negativitet til prosessrelatert arbeid. Den aktuelle 
dagen skulle det gjennomføres en frivillig spørreundersøkelse som Nina valgte å ikke delta på. Dette var ikke første 
spørreundersøkelsen hun hadde latt være å delta på og det førte til reaksjoner blant andre gruppemedlemmer. 
Dette kom kort tid etter en landsbydag hvor Nina hadde nektet deltakelse i grupperefleksjon (Viktig å få med 
grunnen og Ninas synspunkter). Med Ninas utsagn mot grupperefleksjon, som ``tortur'', ble det i meste laget
negativt for Kristoffer og Christoffer. Kristoffer sørget for at temaet ble tatt opp med den hensikt å løse eventuelle
konflikter og skape et bedre gruppemiljø. 

\section{Refleksjoner og aksjoner}

% Negativitet spres veldig i en gruppe
% Ulike synspunkter
% Todelingen

\begin{quote}``
I dag igjen fikk vi utdelt spørreundersøkelser. Vet at også dette har blitt reflektert over tidligere siden personen ikke 
ville svare på spørsmålene den gang heller... 
''\end{quote} \rightline{{\rm --- Kristoffer}}


Todelingen med prosessdel og prosjektdel i faget EiT har vært ganske tydelig i vår gruppe. Det er ikke 
nødvendigvis unaturlig med tanke på at samtlige gruppemedlemmer har mindre erfaring med et slikt gruppearbeid 
og grupperefleksjon. Det kommer også frem i Erlends refleksjoner, som gjengitt under:
\begin{quote}``
Jeg synes det er gøy og produktivt å jobbe sammen med Nina på prosjektdelen, men det skjærer seg litt når vi 
arbeider med prosessdelen. Jeg synes det er veldig demotiverende når et av gruppemedlemmene melder seg ut 
og ikke ønsker å reflektere. Stemningen i etterkant synes jeg har vært anspent.
''\end{quote} \rightline{{\rm --- Erlend}}
Mye av meningen med EiT er å styrke studentenes erfaring innen nettopp gruppesamspillet og det er gjennom
slike situasjoner at man kan ta lærdom. 

Som Erlend også nevner fører dette til at gruppen er mer anspent, noe som er med å bryte ned samarbeidet i gruppen. 

\begin{quote}``
Jeg føler av vi er en gruppe på fem medlemmer og én utenfor. Det er veldig kjipt.
Vi er ikke direkte med hverandre og pakker ting inn i vennlighet. Kristoffer er flinkest av oss til å være direkte.
Jeg har ikke forståelse for hvorfor det er blitt slik eller hvordan den kan bli slik i en gruppe. Dog skjønner jeg at 
Ninas holdninger er mye av grunnen. Resten av gruppa har et veldig bra forhold, etter mitt inntrykk.
''\end{quote} \rightline{{\rm --- Christoffer}}

\begin{quote}``
Jeg syns at det er litt spent stemning i gruppa. Jeg syns at det burde være mer åpenhet og at folk sier hva de 
mener og at folk tør å snakke. Jeg tør ikke å snakke i sånne situasjoner fordi jeg syns det er kleint å ta opp og det 
er slitsomt og vanskelig å forklare hva jeg mener, spesielt uten å virke nedtalende til en annen person, så jeg bare 
lar være. Jeg syns at vi burde gjøre noe for å inkludere Nina mer, siden hun prøver å distansere seg fra gruppen. 
Det at hun ikke er med i fellesoppgaver er ikke greit, jeg syns at alle burde delta i både refleksjon og 
fellesoppgaver. Vi tok det opp en gang tidligere, men jeg føler ikke at vi har fulgt det opp bra nok. Jeg syns at vi bør 
ta tak i ting med en gang det skjer, selv om det kan være ubehagelig.
''\end{quote} \rightline{{\rm --- David}}

\begin{quote}``
Kan begynne med å nevne at jeg kan igrunn forstå det med å ikke ville ta den spørreundersøkelsen, siden den 
tross alt var valgfri og vi fikk ikke vite første gangen at den kom til å gi oss feedback senere. Folk forsiktig med å si 
hva de mener og håper at det blir mindre misnøye og en bare holder kjeft. Vi burde være mer ærlig, og konfrontere 
situasjoner i stedet for å bli passiv. Jeg kan si at jeg personlig er veldig unnvikende og passiv når det er situasjoner 
og elefanter i rommet, og dette er noe jeg må jobbe med personlig.
''\end{quote} \rightline{{\rm --- Emil}}

\begin{quote}``
Jeg synes det er gøy og produktivt å jobbe sammen med Nina på prosjektdelen, men det skjærer seg litt når vi 
arbeider med prosessdelen. Jeg synes det er veldig demotiverende når et av gruppemedlemmene melder seg ut 
og ikke ønsker å reflektere. Stemningen i etterkant synes jeg har vært anspent. Jeg vet ikke helt hvordan jeg skal 
forholde meg til det og blir derfor ganske stille. Kanskje stemningen hadde lettet om vi hadde pratet ut om det.
''\end{quote} \rightline{{\rm --- Erlend}}

\begin{quote}``
Jeg reflekterte over at en person nektet å reflektere og “sier det er som tortur å jobbe med dette” forrige 
landsbydag.  Der skrev jeg: «Synes det begynner å bli nok. Er rett og slett ganske lei av det her. Det skaper et 
elendig samhold i gruppa, stemningen blir ødelagt og motivasjonen for å jobbe sammen med denne personen 
forsvinner. Dette faget går ut på å lære seg hvordan man skal jobbe sammen i en slags simulert arbeidssituasjon, 
og motviljen til å prøve å lære seg dette er irriterende til de grader. Det bør kunne gå an å oppføre seg litt mer 
«profesjonelt». Kommentarer som at «vi kommer uansett ikke til å se hverandre igjen etter EiT er ferdig» og «vi har 
forskjellige interesser og jeg har vil ikke være her med dere» hører ikke hjemme i et sånt samarbeid. Vi har gjort det 
vi kan for å tilpasse oss ettersom vi har tatt opp problemet ved tidligere anledning, men kan ikke se at det har 
skjedd noen endring..»

I dag igjen fikk vi utdelt spørreundersøkelser. Vet at også dette har blitt reflektert over tidligere siden personen ikke 
ville svare på spørsmålene den gang heller. Ettersom det er et såpass sære reaksjoner fra denne personen på 
forskjellige aspekter av prosessarbeidet i EiT burde hun krysset av slik at vi fikk ut reelle svar på undersøkelsen. 
Under møtet med Sofus og co synes jeg også det er leit å høre han si «vi har oppfattet at denne gruppen har agert 
sterkt mot prosessarbeidet» ettersom det kun er denne personen som gjør dette, ikke resten av gruppen. Synes 
også situasjonene burde blitt tatt opp, ettersom det har mye med prosessbiten av faget å gjøre. Når det gjelder 
resultatet av undersøkelsen, ble det med en gang himlet med øynene og så vidt kikket på arket før det ble lagt fra 
seg igjen. Vi burde diskutert det sammen, ikke bare lagt det fra oss.

Synes kontinuerlig kommunikasjon i arbeidet er noe som er nødvendig for at et team skal fungere optimalt. En 
gruppe bør være mer effektiv enn hvert enkelt individ er alene, noe som krever at man kommuniserer godt etter min 
mening.
''\end{quote} \rightline{{\rm --- Kristoffer}}

\begin{quote}``
Skjønner ikke at unnvikelse av å delta på spørreundersøkelse blir sett på som negative holdninger mot gruppa. 
Tortur kommentar: Bruker kanskje kraftige ord, men mener ikke at det er helt uholdbart. Men hun må finne seg i å 
Sists gang: Følte seg stressa for emne som skulle reflekteres om pga fraværet på gruppa. 
Grunnen til bra arbeid sist var pga at ch var borte. Det blir da litt mye tomprat. 
Føler at hun er forskjellig fra resten av gruppa. Hun trekker seg litt ut hun og. 
Likegyldigheten min kommer fra at jeg ikke har betraktet en case for en case. 
''\end{quote} \rightline{{\rm --- Nina}}

\section{Evaluering av aksjoner}
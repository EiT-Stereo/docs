\chapter{Tema 2: Holdinger til gruppe{-}\\refleksjon}

Under følgende tema vil det diskuteres gruppens og enkeltpersoners holdninger til prosessdelen av EiT og hvordan
disse har påvirket samarbeidet i gruppen. Det vil legges frem hvordan tydeligheten av todelingen med prosess og 
prosjekt kom fram i vårt samarbeid og hva som var læringsutbytte fra situasjonen. 

\section{Situasjonsbeskrivelse}

Temaet ble tatt opp når Nina ble konfrontert med sin gjentagende negativitet til prosessrelatert arbeid. Den aktuelle 
dagen skulle det gjennomføres en frivillig spørreundersøkelse som Nina valgte å ikke delta på. Dette var veldig upopulært
blant resten av gruppen ettersom det var ønskelig å lære noe om gruppen ut i fra undersøkelsen. Da hendelsen inntraff 
ble ikke disse reaksjonene tatt opp på grunn av usikkerhet. Dette temaet var et  såkalt ikke- diskuterbart tema, veldig relevant for gruppens oppgave, samtidig som flere i gruppen ikke trodde det var mulig å diskutere saken uten at det fikk negative konsekvenser. \cite{Artikkel3} Kristoffer sørget for at temaet ble tatt opp med den hensikt å løse eventuelle
konflikter og skape et bedre gruppemiljø.

\section{Refleksjoner og aksjoner}

Etter dagens situasjon var det nødvendig å få belyst saken og få bedret stemningen i gruppa. Det førte til en lang
refleksjon hvor samtlige lærte mye om seg selv og de andre på gruppa. Det kom også frem at todelingen med prosessdel
og prosjektdel var tydelig i vår gruppe. Med det menes at teamet jobbet bra under prosjektarbeid, men ulike holdninger
til prosessdel førte til en splittelse av teamet. 

\begin{quote}``
Jeg føler av vi er en gruppe på fem medlemmer og ett utenfor. Det er veldig trist...
''\end{quote} \rightline{{\rm --- Christoffer}}

Christoffer syntes at det var trist at teamet ikke kunne ha det bedre sammen. Det begynte også å få konsekvenser
for prosjektarbeidet. Å avslutte en effektiv arbeidsdag med negativitet satte preg på de kommende landsbydagene.

\begin{quote}``
Jeg synes det er gøy og produktivt å jobbe sammen med Nina på prosjektdelen, men det skjærer seg litt når vi 
arbeider med prosessdelen. Jeg synes det er veldig demotiverende når et av gruppemedlemmene melder seg ut 
og ikke ønsker å reflektere. Stemningen i etterkant synes jeg har vært anspent.
''\end{quote} \rightline{{\rm --- Erlend}}

Mye av meningen med EiT er å styrke studentenes erfaring innen nettopp gruppesamspillet og det er gjennom
slike situasjoner at man kan ta lærdom. 
Som Erlend også nevner fører dette til at gruppen er mer anspent, noe som er med å bryte ned samarbeidet i 
gruppen. Anspentheten Erlend referer til skinner veldig gjennom hos gruppemedlemmene og Kristoffers refleksjoner 
er veldig beskrivende:

\begin{quote}``
Jeg reflekterte over at en person nektet å reflektere og «sier det er som tortur å jobbe med dette» forrige 
landsbydag. Der skrev jeg: «Synes det begynner å bli nok. Er rett og slett ganske lei av det her. Det skaper et 
elendig samhold i gruppa, stemningen blir ødelagt og motivasjonen for å jobbe sammen med denne personen 
forsvinner. Dette faget går ut på å lære seg hvordan man skal jobbe sammen i en slags simulert arbeidssituasjon, 
og motviljen til å prøve å lære seg dette er irriterende til de grader. Det bør kunne gå an å oppføre seg litt mer 
profesjonelt. Kommentarer som at «vi kommer uansett ikke til å se hverandre igjen etter EiT er ferdig» og «vi har 
forskjellige interesser og jeg vil ikke være her med dere» hører ikke hjemme i et sånt samarbeid. Vi har gjort det 
vi kan for å tilpasse oss ettersom vi har tatt opp problemet ved tidligere anledning, men kan ikke se at det har 
skjedd noen endring..» 
''\end{quote} \rightline{{\rm --- Kristoffer}}

Opp til dette tidspunktet har flere gruppemedlemmer skjøvet bort problemet som holdt på å oppstå, men det var også 
ulike synspunkter angående problemet.

\begin{quote}``
Skjønner ikke at unnvikelse av å delta på spørreundersøkelse blir sett på som negative holdninger mot gruppa. 
Tortur kommentaren var kanskje kraftige ord, men mener ikke at det er helt uholdbart. Da det på et tidligere tidspunkt skulle
reflekteres rundt det effektive samarbeidet i Christoffers fravær følte jeg meg stressa for emne det skulle reflekteres om.
''\end{quote} \rightline{{\rm --- Nina}}

\begin{quote}``
Jeg kan igrunn forstå det med å ikke ville ta den spørreundersøkelsen, siden den 
tross alt var valgfri og vi fikk ikke vite første gangen at den kom til å gi oss tilbakemelding senere.
''\end{quote} \rightline{{\rm --- Emil}}

Gruppen består av medlemmer med forskjellig bakgrunn, og i følge Johnson og Johnson (2013) så har mangfoldet i en gruppe 
både styrker og svakheter. Folk med forskjellig bakgrunn oppfatter ting forskjellig og misforståelser og ulike synspunkter er 
ikke uvanlig. Misforståelser er ikke nødvendigvis sunt for et bra samarbeid, men ulike synspunkt kan føre til mer kreativitet 
og produktivitet. \cite{Artikkel2} Synspunktene til de enkelte kan bidra til en videre forståelse av et problem, men det kan også oppstå 
konflikter som denne situasjonen er et eksempel på.
For å unngå slike konflikter som følger av at personer sitter inne med meninger bør personen ta det opp umiddelbart.

\begin{quote}``
Jeg syns at det burde være mer åpenhet og at folk sier hva de 
mener og at folk tør å snakke. Jeg tør ikke å snakke i sånne situasjoner fordi jeg syns det er kleint å ta opp og det 
er slitsomt og vanskelig å forklare hva jeg mener, spesielt uten å virke nedlatende til en annen person. Jeg lar derfor
det bare være. Jeg syns at vi burde gjøre noe for å inkludere Nina mer, siden hun prøver å distansere seg fra gruppen. 
Det at hun ikke er med i fellesoppgaver er ikke greit, jeg syns at alle burde delta i både refleksjon og 
fellesoppgaver. Vi tok det opp en gang tidligere, men jeg føler ikke at vi har fulgt det opp bra nok. Jeg syns at vi bør 
ta tak i ting med en gang det skjer, selv om det kan være ubehagelig.
''\end{quote} \rightline{{\rm --- David}}

David nevner at det aktuelle temaet har blitt tatt opp tidligere, men uten ønsket effekt. Da dette sist ble tatt opp ble
det tatt opp med en veldig vennlig tone. Dette kan påvirke uttrykket til den enkelte og den eksakte meningen 
forsvinner bak all vennligheten, noe som er et vanlig problem for mangfoldige grupper. \cite{Artikkel2} 
Kommunikasjon i en gruppe må være mer presis og meninger må komme frem på
riktig vis dersom det skal læres av samarbeidet.  

\begin{quote}``
Ettersom det er et såpass sære reaksjoner fra denne personen på 
forskjellige aspekter av prosessarbeidet i EiT burde hun krysset av slik at vi fikk ut reelle svar på undersøkelsen. 
Under møtet med de EiT ansvarlige synes jeg også det er leit å høre han si «vi har oppfattet at denne gruppen har agert 
sterkt mot prosessarbeidet» ettersom det kun er denne personen som gjør dette, ikke resten av gruppen. Synes 
også situasjonene burde blitt tatt opp, ettersom det har mye med prosessbiten av faget å gjøre. Når det gjelder 
resultatet av undersøkelsen, ble det med en gang himlet med øynene og så vidt kikket på arket før det ble lagt fra 
seg igjen. Vi burde diskutert det sammen, ikke bare lagt det fra oss.

Synes kontinuerlig kommunikasjon i arbeidet er noe som er nødvendig for at en gruppe skal fungere optimalt. En 
gruppe bør være mer effektiv enn hvert enkelt individ er alene, noe som krever at man kommuniserer godt etter min 
mening.
''\end{quote} \rightline{{\rm --- Kristoffer}}

Som Kristoffer nevner må alle kortene på bordet for at gruppen skal fungere optimalt. To definerende faktorer
for en gruppe er gjensidig avhengighet og felles ansvar \cite{Artikkel4}, som betyr at man er avhengig av at
alle bidrar. Felles ansvar kan tolkes som at gruppen har samme ansvar, men også at ansvaret er fordelt. I slike 
grupper er det viktig med god kommunikasjon for å kunne samsvare det fordelte ansvaret og utnytte avhengigheten.
På grunn av dette så er det viktig at alle deltar og føler seg som en del av gruppen.

Dagens situasjon førte til en av våre mest åpne og ærlige refleksjoner. Det var veldig tydelig at
alle ønsket og bedre situasjonen og at dette var viktig for at vi skulle kunne fullføre EiT. 

\section{Evaluering av aksjoner}

Etter at vi hadde reflektert over temaet ble vi mer oppmerksomme på viktigheten av å ta opp konflikter underveis.
Kommunikasjonen må være mer tydelig slik at problemet kan komme frem i sin helhet og løses. 
Vi forsto også at alles meninger må komme frem dersom det skal kunne gjøres en grundig analyse av situasjonen
og dersom det skal være læringsrikt. 

Det var veldig viktig at Kristoffer tok på seg ansvaret slik at vi fikk belyst det voksende problemet.
Det viser også en sterk egenskap når et gruppemedlem som blir konfrontert av flere gruppemedlemmer er villig til 
å samarbeide for å løse problemet. 
Som vi alle sikkert forstod, men ikke turte, var eneste løsningen på problemet å snakke om det. 
Med tanke på hvor effektiv gruppen kunne være under prosjektarbeid var det dumt å la prosessdelen påvirke
denne effektiviteten. Ganske umiddelbart etter denne refleksjonen ble samholdet i gruppen bedre,
overgangen fra prosjektarbeid til prosessarbeid var nå blitt mye mindre, og vi jobbet mer som en fungerende gruppe.

Seks gruppemedlemmer kan nå etter et fire måneder langt samarbeid stå samlet som en gruppe når det hele avsluttes.
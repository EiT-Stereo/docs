\chapter{Oppsummering (skift tittel)}

\section{Individuelle Refleksjoner}

\subsection*{Christoffer Ramstad-Evensen}

Etter å ha fullført EiT er inntrykkene jeg sitter igjen med positive. Ja, vi har innad i gruppen hatt våre utfordringer
på veien til hit vi er i dag, men totaltsett har vi fullført dette faget sammen. Når jeg sier dette er det for å rette 
fokuset på at læringsutbytte i denne perioden har vært stort. Noen elementer i læringen som har funnet sted
har vært selvsagte, men av og til trenger man å få det repetert. I tillegg er det som jeg nevnte innledningsvis
i introduksjonen en erfaring ved å jobbe i et team som man ikke kan få nok av. Jeg har også lært noen 
definisjoner jeg ikke skulle vært foruten samt forstått viktigheten av og ha en form for ledeles i et team. 
Vi har under hele EiT jobbet som et team uten leder, og det har gått greit, men arbeidet blir først fremprovosert
når noen tar ansvar og setter prosessen i gang. 

I tidlig fase av EiT syntes jeg oppsettet var ganske kunstig. Det var en situasjon jeg ikke kunne se for meg
finne sted i et reelt arbeidsmijlø. Vi hadde hele tiden veldig fokus på å evaluere hver eneste tanke og ord som
skulle deles i fellesskap. Dette førte til en del unødvendige småkonflikter og kommunikasjonen ble lite direkte. 
Likevel tror jeg at det må være kunstig på denne måten for å fremprovosere situasjoner man kan ta lærdom av.

Jeg føler jeg tar med med mye positivitet og gode erfaringer fra dette faget og stiller sterkere i teamarbeid hos
fremtidige arbeidsgivere.

\subsection*{Kristoffer Løvall}
Det som har skilt seg ut med dette gruppearbeidet for min del, er at vi endte opp som en gruppe med flat struktur. Jeg hadde som mål i starten av 
semesteret å holde meg litt unna lederrollen, og synes det har fungert overraskende godt. I tillegg til at jeg selv pleier å ta på meg en lederrolle, har 
jeg oppfattet både Christoffer og til tider Nina som naturlige ledertyper, og ettersom vi er flere i gruppen med disse egenskapene, føltes det helt 
naturlig å fordele alle beslutninger og alt ansvar jevnt. Jeg har merket at det å ha et noe mer avslappet forholdt til gruppearbeid enn tidligere har 
fungert godt med tanke på effektivitet, samt det å stole på en gruppe mennesker jeg nettopp har møtt til å fullføre en oppgave på en 
tilfredsstillende måte. Dette kan gjerne være fordi gruppen har holdt et veldig høyt faglig nivå, og ettersom jeg er den eneste på gruppen med 
maskinfaglig bakgrunn har jeg ikke hatt andre personer å “sparre” med innen mitt fagfelt. Som tidligere nevnt har jeg vært klar over at jeg kan bli 
noe kontrollerende i engasjerte øyeblikk, men denne (dårlige) egenskapen har jeg klart å holde i sjakk under arbeidet med EiT. Et annet aspekt som 
har overrasket positivt er selve refleksjonsbiten. Det var uten tvil en utfordring for meg, og andre i gruppen, å dele tanker og meninger på en 
såpass åpen måte til ukjente mennesker i begynnelsen. Etter hvert som vi kom litt mer inn i det, har det vist seg at det, ikke bare i dette 
gruppearbeidet, men også i andre parallelle fag med gruppearbeid, har skapt et mye bedre sammarbeid på et faglig så vel som sosialt plan. Jeg vil 
uten tvil ha et større fokus på meg selv for kommende gruppearbeid, altså på hvordan jeg oppfører meg ovenfor gruppen, og mindre på hvordan 
gruppen oppfører seg ovenfor meg.



\section{Grupperefleksjoner}
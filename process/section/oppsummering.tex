\chapter{Oppsummering (skift tittel)}
I dette kapittelet oppsummeres våre efaringer fra prosessdelen av EiT.
Vi går gjennom både individuelle- og grupperefeleksjoner.

\section{Individuelle Refleksjoner}
De individuelle refleksjonene forteller om erfaringene hvert enkelt gruppemedlem
har gjort seg gjennom dette halvåret.

\subsection*{Christoffer Ramstad-Evensen}
Etter å ha fullført EiT er inntrykkene jeg sitter igjen med positive. Ja, vi har innad i gruppen hatt våre utfordringer
på veien til hit vi er i dag, men totaltsett har vi fullført dette faget sammen. Når jeg sier dette er det for å rette 
fokuset på at læringsutbytte i denne perioden har vært stort. Noen elementer i læringen som har funnet sted
har vært selvsagte, men av og til trenger man å få det repetert. I tillegg er det som jeg nevnte innledningsvis
i introduksjonen en erfaring ved å jobbe i et team som man ikke kan få nok av. Jeg har også lært noen 
definisjoner jeg ikke skulle vært foruten samt forstått viktigheten av og ha en form for ledeles i et team. 
Vi har under hele EiT jobbet som et team uten leder, og det har gått greit, men arbeidet blir først fremprovosert
når noen tar ansvar og setter prosessen i gang. 

I tidlig fase av EiT syntes jeg oppsettet var ganske kunstig. Det var en situasjon jeg ikke kunne se for meg
finne sted i et reelt arbeidsmijlø. Vi hadde hele tiden veldig fokus på å evaluere hver eneste tanke og ord som
skulle deles i fellesskap. Dette førte til en del unødvendige småkonflikter og kommunikasjonen ble lite direkte. 
Likevel tror jeg at det må være kunstig på denne måten for å fremprovosere situasjoner man kan ta lærdom av.

Jeg føler jeg tar med med mye positivitet og gode erfaringer fra dette faget og stiller sterkere i teamarbeid hos
fremtidige arbeidsgivere.

\subsection*{Nina Margrethe Smørsgård}
Etter å ha jobbet med EiT i et halvår nå, sitter jeg igjen med blanda følelser. 
Jeg merker at jeg liker godt å jobbe målretta med et håndfast prosjekt, men jeg 
har også merket at jeg har hatt veldig lite tålmodighet med prosessdelen. For 
min del ble det arbeidsoppgavene i begynnelsen veldig vage, og det ble veldig passivt.

Jeg er av natur en person som ikke liker å tenke og føle så mye på det som skjer 
rundt meg, men heller jobber godt med håndfaste ting. Det betyr ikke at jeg ikke ser 
nytten av refleksjoner og evalueringer slik det er blitt gjort i EiT, jeg ser det 
bare som en veldig unaturlig og kunstig måte å tvinge det fram på, og dermed ser 
jeg ikke hvordan det kan overføres til en reell arbeidssituasjon.

Jeg tror grunnen til at jeg etterhvert ble så negativ til prosessdelen av EiT 
er at det ble så mye prosess i starten, men jeg vet ikke om det hadde vært bedre å få 
det fordelt litt og litt utover hele semesteret. Generelt sett mener jeg det er 
bedre å rive av plasteret ``fort og gæli'', og det gjør vondt der og da, men 
da der det også mye raskere over. 

På den annen side har jeg lært at folk forstår ting på forskjellig måte ut fra 
hvilken bakgrunn de har, og hvilken type kommunikasjon de er vante med fra før.
Jeg har en tendens til å kaste ut ganske harde ord og uttrykk, og har nok 
derfor blitt misforstått som noe mer negativ enn jeg faktisk har vært. 


\section{Grupperefleksjoner}
\chapter{Oppsummering (skift tittel)}

\section{Individuelle Refleksjoner}

\subsection*{Christoffer Ramstad-Evensen}

Etter å ha fullført EiT er inntrykkene jeg sitter igjen med positive. Ja, vi har innad i gruppen hatt våre utfordringer
på veien til hit vi er i dag, men totaltsett har vi fullført dette faget sammen. Når jeg sier dette er det for å rette 
fokuset på at læringsutbytte i denne perioden har vært stort. Noen elementer i læringen som har funnet sted
har vært selvsagte, men av og til trenger man å få det repetert. I tillegg er det som jeg nevnte innledningsvis
i introduksjonen en erfaring ved å jobbe i et team som man ikke kan få nok av. Jeg har også lært noen 
definisjoner jeg ikke skulle vært foruten samt forstått viktigheten av og ha en form for ledeles i et team. 
Vi har under hele EiT jobbet som et team uten leder, og det har gått greit, men arbeidet blir først fremprovosert
når noen tar ansvar og setter prosessen i gang. 

I tidlig fase av EiT syntes jeg oppsettet var ganske kunstig. Det var en situasjon jeg ikke kunne se for meg
finne sted i et reelt arbeidsmijlø. Vi hadde hele tiden veldig fokus på å evaluere hver eneste tanke og ord som
skulle deles i fellesskap. Dette førte til en del unødvendige småkonflikter og kommunikasjonen ble lite direkte. 
Likevel tror jeg at det må være kunstig på denne måten for å fremprovosere situasjoner man kan ta lærdom av.

Jeg føler jeg tar med med mye positivitet og gode erfaringer fra dette faget og stiller sterkere i teamarbeid hos
fremtidige arbeidsgivere.

\section{Grupperefleksjoner}
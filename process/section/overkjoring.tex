\chapter{Tema 4: Overkjøring i\\diskusjoner}
Det fjerde temaet vi har valgt er overkjøring i diskusjoner.
Dette temaet handler mer om felles interesse for innsikt i alle deler av
prosjektet, enn faktisk overkjøring av enkeltpersoner under gruppediskusjon.

\section {Situasjonsbeskrivelse}
Fjerde landsbydag sitter Christoffer, Kristoffer og Emil å diskuterer hvordan de fysiske egenskapene til boltene kan måles, samples og så deles opp i pakker for å sende på nettverket til en lastebil. Diskusjonen er intens: 

\begin{quote}``
Hvis vi antar frekvensen vi leter etter ikke er lavere enn 20 Hz kan vi sample så kort som 50 ms uten tap av data. Med en samplingshastighet på 20kHz og en oppløsning på 12 bit blir ikke pakkene større enn 1.5kB, og med en slik sampling hver tiende sekund burde det ikke by på noen problemer for nettverkskapasiteten.
''\end{quote} \rightline{{\rm --- Emil}}

\begin{quote}``
Men hvis vi skal sample med så lav samplingshastighet er vi avhenging av et andreordens lavpassfilter på sensoren.
''\end{quote} \rightline{{\rm --- Christoffer}}

\begin{quote}``
Hvis vi sampler så sjelden, er vi ikke da avhengig av å ha noen form for aktuator som garanterer at akslingen vil gi fra seg signaler hver gang?
''\end{quote} \rightline{{\rm --- Kristoffer}}

Diskusjonen fortsetter i 5-10 minutter mens d tre andre i gruppen sitter bare å hører på. Ingen av de kommer med innspill eller tanker om hva som blir diskustert, og etterhvert som diskusjonen går detter David av  og begynner jobbe med sitt eget. Nina og Erlend følger med, men føler ikke de har noe å komme med. 

\begin{quote}``
Problemstillingen er mekatronikk-preget. Det er ikke akkurat rart at de som er direkte påvirket av denne teknologien diskuterer grundigst mens andre lytter. Jeg synes det er kjempespennende å høre på og jeg kunne fortsatt å lytte enda lenger for å lære om emnet.
''\end{quote} \rightline{{\rm --- Nina}}

\begin{quote}``
Diskusjonen var viktig, og jeg følte det var litt uten for mitt fagfelt. Jeg syntes det var greit å lytte til samtalen og tenkte egentlig ikke så mye på at jeg var passiv.
''\end{quote} \rightline{{\rm --- Erlend}}

Alle diskusjoner i gruppen hadde fram til denne dagen blitt tatt på denne måten; hele gruppen deltok, enten aktivt, eller bare ved å høre på. Dette var i hvertfall skil det ble oppfattet før dagens refleksjoner. Det ble tatt opp om det kanskje ikke var den mest effektive måten å føre diskusjoner på, selv om alle i gruppen ville lære mest mulig. Et av underkapitellene i sammarbeidsavtalen vår er nettopp læring, og dette var en av hovedgrunnene til at de fleste diskusjoner ble tatt i plenum slik at alle i gruppen skulle få innsikt og læring innen mest mulig av prosjektet.

\section{Refleksjoner og aksjoner}
Som de fleste refleksjonene ble det også på denne satt av 10 minutter hvor hver person skrev ned sine tanker og refleksjoner om situasjonen, og fikk deretter mulighet til å presentere sine tanker hver for seg. Etter individuelle refleksjoner ble det reflektert i fellesskap.

Emil begynte med å si at han var den som påpekte da dette skjedde fordi han følte det kanskje var litt overkjøring ute å gikk. Han synes diskusjonen var produktiv selv for seks personer, men at det ikke burde være slik at det er tre personer som styrer skuten mens de tre andre ikke har noe å si. Christoffer var veldig enig i dette, men følte også at de som var passive i samtalen kanskje burde tatt til ordet før Emil om at de følte seg utenfor. De tre som var passive hadde noenlunde samlet mening om at diskusjonen var interessant, og det var litt av grunnen til de ikke grep inn, men i ettertid satt de kanskje litt igjen med det inntrykket produktiviteten var litt lav. De følte derimot ikke at de var overkjørt, det var bare ikke noe de hadde noe å si noe om.

Under fellesrefleksjonene kom det fram stor enighet at dette var et problem som ville løse seg om en delte opp prosjektet i underkategorier hvor man heller jobbet i grupper på to og to, og de fleste diskusjonene ble da tatt innad i underkategorigruppene. En kategori i samarbeidsavtalen var læring, og det ble derfor diskutert at eventuelle tap av læring på grunn av den nye gruppeinndelingen, kunne gjøres opp ved å ha små erfaring- og kunnskapsutevekslingssamtaler når det hadde blitt gjort framskritt innen de enkelte undergruppene.

Resultat av denne casen ble derfor at gruppen delte seg opp i mindre grupper ut ifra de respektive studiene til medlemmene. Dette var også noe som kom mer naturlig etterhvert som prosjektarbeid begynte å gå framover, siden det fram til denne landsbydagen hadde gått bort mye tid i øvelser og teambuilding.

\section{Evaluering av aksjoner}
Å dele opp gruppen viste seg å være et godt valg. De fleste diskusjoner som dukket opp etter den nye gruppeinndelingen ble som oftest tatt innad i undergruppene, samtidig som diskusjoner og problemer som spredde seg over flere disipliner ble tatt i plenum, noe som Schwarz (2002) mener er hvordan en effektiv gruppe burde løse problemer. \cite{Artikkel3} Å dele opp i undergrupper kom også med flere fordeler, arbeidsoppgavene i gruppen ble oppdelt med mer konkrete skiller, slik at hver enkelt undergruppe fikk sitt klare område å jobbe med. Når disse biten så skal settes sammen er det lettere å se hvem som har gjort en innsats, og hvem som ikke har gjort det. Gruppen blir da mer gjensidig avhengig av hverandre og det er ikke lenger enkeltpersoner som kan dra hele lasset. Dette, kombinert med at resultatet til gruppen blir vurdert i en helhet, og ikke på enkeltpersoners bidrag, som da betyr et høgt felles ansvar, gjør at gruppen havner i team kategorien. \cite{Artikkel4}


Det er rimelig å anta at en slik inndeling i undergrupper ville kommet før eller senere, uavhengig av akkurat den situasjonen som utløste det, men at den kom før heller en senere var en positiv ting som resulterte i mer konkrete arbeidsoppgaver, mindre avbrudd og mer effektive diskusjoner siden at de som ikke hadde full innsikt trengte å bli forklart detaljer som virker åpenbare for de med innsikt i den gitte disiplinen.
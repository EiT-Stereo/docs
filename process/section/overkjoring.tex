\chapter{Tema 4: Overkjøring i diskusjoner}
Det fjerde temaet vi har valgt er overkjøring i diskusjoner.
Dette temaet handler mer om felles interesse for innsikt i alle deler av
prosjektet, enn faktisk overkjøring av enkeltpersoner under gruppediskusjon.

\section {Situasjonsbeskrivelse}
Frem til 28. januar brukte gruppen å diskutere de aller fleste situasjoner i plenum. En av sakene som ble diskutert i plenum handlet om samspillet mellom mekanikken med sensorer, med videre oppkobling til mikrokontrollere, og hvilke fysiske egenskaper som kunne måles for at vi skulle kunne løse vår problemstilling. Siden de fleste situasjoner ble tatt i plenum ble denne også denne det. Problemet var at dette var en sak som bare Christoffer, Kristoffer og Emil hadde noe innsikt i, og noe å si noe om. De andre i gruppen ble da for det meste tause, og satt bare å hørte på. Midt i samtalen ble det derfor tatt opp at dette var kanskje en samtale som burde bare tas under 6 øyne?

\section{Refleksjoner og aksjoner}
Som de fleste refleksjonene ble det også på denne satt av 10 minutter hvor hver person skrev ned sine tanker og refleksjoner om situasjonen, og fikk deretter mulighet til å presentere sine tanker hver for seg. Etter individuelle refleksjoner ble det reflektert i fellesskap.

Emil begynte med å si at han var den som påpekte da dette skjedde fordi han følte det kanskje var litt overkjøring ute å gikk. Han synes diskusjonen var produktiv selv for seks personer, men at det ikke burde være slik at det er tre personer som styrer skuten mens de tre andre ikke har noe å si. Christoffer var mye enig i dette, men følte også at de som var passive i samtalen kanskje burde tatt til ordet før Emil om at de følte seg utenfor. De tre som var passive hadde noenlunde samlet mening om at diskusjonen var interessant, og det var litt av grunnen til de ikke grep inn, men i ettertid satt de kanskje litt igjen med det inntrykket produktiviteten var litt lav. De følte derimot ikke at de var overkjørt, det var bare ikke noe de hadde noe å si noe om.

Under fellesrefleksjonene kom det fram stor enighet at dette var et problem som ville løse seg om en delte opp prosjektet i underkategorier hvor man heller jobbet i grupper på to og to, og de fleste diskusjonene ble da tatt innad i underkategorigruppene. En kategori i samarbeidsavtalen var læring og det ble derfor diskutert at eventuelle tap av læring pga den nye gruppeinndelingen kunne gjøres opp ved å ha små erfaring- og kunnskapsutevekslingssamtaler når det hadde blitt gjort framskritt innen de enkelte undergruppene.

Resultat av denne casen ble derfor at gruppen delte seg opp i mindre grupper ut ifra de respektive studiene til medlemmene. Dette var også noe som kom mer naturlig etterhvert som prosjektarbeid begynte å gå framover, siden det fram til denne landsbydagen hadde gått bort mye tid i øvelser og team-building.

\section{Evaluering av aksjoner}
Å dele opp gruppen viste seg å være god suksess. De fleste diskusjoner som dukket opp etter den nye gruppeinndelingen ble som oftest tatt innad i undergruppene, samtidig som diskusjoner og problemer som spredde seg over flere disipliner ble tatt i plenum, noe som Schwarz (2002) mener er hvordan en effektiv gruppe burde løse problemer. Å dele opp i undergrupper kom også med flere fordeler - arbeidsoppgavene i gruppen ble oppdelt med mer konkrete skiller, slik at hver enkelt undergruppe fikk sitt klare område å jobbe med. Når disse biten så skal settes sammen er det lettere å se hvem som har gjort en innsats, og hvem som ikke har gjort det. Gruppen blir da mer gjensidig avhengig av hverandre og det er ikke lenger enkeltpersoner som kan dra hele lasset. Dette, kombinert med at resultatet til gruppen blir vurdert i en helhet, og ikke på enkeltpersoners bidrag, som da betyr et høgt felles ansvar, gjør at gruppen havner i team kategorien (Hjertø, 2013).

Det er rimelig å anta at en slik inndeling i undergrupper ville kommet før eller senere, uavhengig av akkurat den situasjonen som utløste det, men at den kom før heller en senere var en positiv ting som resulterte i mer konkrete arbeidsoppgaver, mindre avbrudd og mer effektive diskusjoner siden at de som ikke hadde full innsikt trengte å bli forklart detaljer som virker åpenbare for de med innsikt i den gitte disiplinen.
\chapter{Tema 3: Avsporing}

Det tredje temaet vi presenterer handler om avsporing i diskusjoner. Gjennom EiT har gruppen opplevd at diskusjonene
blir ufullstendige på grunn av avsporinger. Dette er ikke så lett å unngå i en såpass stor gruppe, der alle har egne tanker og meninger.
Det er dog viktig at avsporingene ikke blir for store slik at det hindrer fremdrift mot målet. For å unngå dette kan det være lurt
å ha en lederrolle som kan sette foten ned når det blir nok. Teamet vårt har vært et lederløst team og det kan ha ført til at 
enkelte avsporinger har eskalert. Det er sunnere for et team og ha en form for ledelse og et lederløst team er ofte et hinder for 
fremdrift og en kilde til frustrasjon. \cite{Artikkel4}

\section{Sitasjonsbeskrivelse}

Under den tredje landsbydagen var gruppen fortsatt i oppstartsfasen, kalt forming. \cite{forming}\\Denne fasen er preget av at medlemmene er 
litt ekstra høflige og positive. Dette var også tilfellet for vår gruppe hvor enkelte medlemmer ofte stoppet opp for å gi positive 
tilbake{-}\\meldinger, slik som: ``Flott, dette får vi bruk for i rapporten!'', når andre kom opp med ideer og diskusjoner. Spesifikt var det 
denne typen avsporing som enkelte gruppemedlemmer syntes var unødvendig. 

\section{Refleksjoner og aksjoner}

Vi merket tidlig i samarbeidet at gruppen er satt sammen av sterke personligheter som liker å være sosiale. Det ble derfor en god del småprat. I tillegg til dette kunne det merkes at flere hadde et behov for å markere seg som pådriver til godt arbeid. Dette kan beskrives som en slags rollekonflikt \cite{Artikkel2} der Christoffers innspill kan ha blitt oppfattet som et forsøk på å ta kontrollen over arbeidet og etablere seg som en leder. Nina er også en naturlig ledertype og viste dette ved å ta ansvar under refleksjonene ettersom vi hadde blitt enige om en lederløs struktur. Nina som gikk hardest ut mot denne formen for avbrudd og kom med følgende utsagn: 

\begin{quote}``
Føler at det blir en form for selvdigging. Vi prioriterer heller å fortelle om hvor gode vi er til å gjøre noe, i 
stedet for å faktisk gjøre det. Det er klart det er viktig å gi positive tilbakemeldinger, men vi må prøve å holde det på et konstruktivt nivå, slik at vi ikke gir oss selv mestringsfølelse uten å ha oppnådd noe.
''\end{quote} \rightline{{\rm --- Nina}}

Dette utsagnet følte Kristoffer seg noe uenig i og hadde følgende kommentar:

\begin{quote}``
Ser det mer som positivt at det kommer fram at ``dette 
var bra til rapporten'' fordi da vet vi at vi er inne i en god tankegang. Viktig å validere meningene til andre og hjelpe til å motivere, spesielt i oppstartsfasen. 
''\end{quote} \rightline{{\rm --- Kristoffer}} 
Christoffer hev seg fort med på meningene til Kristoffer og velger å legge til at han ikke ser det som at ``vi er best'', men kjenner at han 
blir gira når medlemmer gir positiv feedback. 

Erlend tror mye av grunnen til at vi sporer av er fordi vi ikke helt vet hva vi skal gjøre, og at vi ikke har noen konkret 
arbeidsoppgave. Denne meningen deler David og Emil, og tror at det vil løse seg når gruppen kommer lenger ut i prosjektet og får 
formulert en problemstilling og delegert arbeidsoppgaver. Det kan lett bli litt frusterasjon når en ikke ikke har et klart mål på hva en 
skal gjøre. I følge Schatz, 2002, er det viktig å ha et klart formål med gruppen, en grunn til at den skal eksistere. \cite{Artikkel3}

Gruppen under ett syns det var bra noen tok opp situasjonen, ikke bare fordi avsporing kan føre til lav effektivitet, men også fordi det 
er tydelig at folk har ulik tolkning av situasjoner. Mens Nina synes det virket som gruppen satt å sjøldigget, tolket Christoffer og 
Kristoffer at gruppemedlemmene egentlig bare ville oppmuntre hverandre siden vi ikke kjente hverandre så godt, og vi var ute etter å 
bygge et positivt miljø i gruppen. Dette understreker at det er viktig med lav terskel for å ta opp saker som enkelte kan tolke negativt 
uten at andre gjør det.

Det viktigste gruppen sitter igjen med denne situasjonen er at det burde være lav terskel for å ta opp saker siden folk kan tolke ting så 
ulikt, og vi burde ikke være for redd å pakke ting inn i for mye ``vennlighet'', slik at budskapet blir forvrengt. Angående selve avsporingen 
og at det blir mye prating uten noen notater, burde gruppen bli flinkere til å ta notater under diskusjonene, slik at om viktig poeng 
kommer opp blir de husket. Vi ble også enige om å rekke opp hånden under diskusjoner for at flyten i samtalen skulle bli bedre, og for at dette skulle skape en litt høyere terskel for å bryte ut med urelevante kommentarer under en diskusjon. Tanken bak dette var at det ville være unaturlig å bryte ut med ``dette er god rapportmat'' etter å ha rukket opp hånden og ventet på tur.

\section{Evaluering av aksjoner}

Etter denne landsbydagen merket vi at avsporingen så og si forsvant, i hvertfall for en periode. Christoffer var veldig bevist over sine egne utsagn, og prøvde holde positive tilbakemeldinger så konstruktive som mulig. Vi merket at avgjørelsene ble tatt på bedre grunnlag ettersom folk fikk slippe til med innspillene sine, og at mer ble gjort i løpet av sammarbeidet. Noe av dette kan også skyldes at oppgaven ble mer konkret og enklere å jobbe med rundt denne tiden, men vi merket at håndsopprekkningen hadde en god effekt på kvaliteten og utbyttet av diskusjonene. En bieffekt ved håndsopprekningen viste seg også å være at gruppemedlemmene tok arbeidet mer seriøst. Kommentarene og inspillene ble rettet kun mot arbeidet og oppgaven, og det sosiale ble for det meste konsentrert i lunsjpausene. Selv om vi ikke klarte holde på rutinene med håndsopprekkning, var det klart at vi hadde tillagt oss en vane med å konsentrere oss om oppgaven og redusert avsporingen til et nivå som var langt bedre enn tidligere.

Angående refleksjonene på slutten av landsbydagene merket vi at terskelen for å ta opp slike temaer ble veldig mye lavere, noe som har ført til mer åpenhet i gruppen og et bedre sammarbeid. Gruppemedlemmene ser i etterkant ut til å ha lettere for å komme med kommentarer, og slipper å sitte inne med ting hele dagen før de tør å ta det opp.



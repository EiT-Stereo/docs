\chapter{Tema 3: Avsporing}

Det tredje temaet vi presenterer handler om avsporing i diskusjoner. Gjennom EiT har teamet opplevd at diskusjonene
blir ufullstendige på grunn av avsporinger. Dette er ikke så lett å unngå i et så stort team hvor alle har tanker og meninger.
Det er dog viktig at avsporingene ikke blir for store slik at det hindrer fremdrift mot målet. For å unngå dette kan det være lurt
å ha en lederrolle som kan sette foten ned når det blir nok. Teamet vårt har vært et lederløst team og det kan ha ført til at 
enkelte avsporinger har eskalert. Det er sunnere for et team og ha en form for ledelse og et lederløst team er ofte et hinder for 
fremdrift og en kilde til frustrasjon \cite{Artikkel4}.

\section{Sitasjonsbeskrivelse}

Under den tredje landsbydagen var gruppen fortsatt under oppstartsfasen, kalt forming \cite{forming}. Denne fasen er preget av at medlemmene er 
litt ekstra høflige og positive. Dette var også tilfellet for vår gruppe hvor enkelte medlemmer ofte stoppet opp for å gi positive 
tilbakemeldinger, slik som: ``Flott, dette får vi bruk for i rapporten!'', når andre kom opp med ideer og diskusjoner. Spesifikt var det 
denne typen avsporing som enkelte gruppemedlemmer syntes var unødvendig.

\section{Refleksjoner og aksjoner}

 Under denne dagens refleksjon var Nina  som gikk hardest ut mot denne formen for avbrudd og kom med følgende utsagn: 
\begin{quote}``
Føler at det blir en form for selvdigging. Vi prioriterer heller å fortelle om hvor gode vi er til å gjøre noe, i 
stedet for å faktisk gjøre det.
''\end{quote} \rightline{{\rm --- Nina}}

Dette utsagnet følte Kristoffer seg sterkt uenig i og hadde følgende kommentar:
\begin{quote}``
Ser det mer som positivt at det kommer fram at ``dette 
var bra til rapporten'' fordi da vet vi at vi er inne i en god tankegang. Viktig å validere meningene til andre og hjelpe til å motivere. 
''\end{quote} \rightline{{\rm --- Nina}} 
Christoffer hev seg fort med på meningene til Kristoffer og velger å legge til at han ikke ser det som at vi er best, men kjenner at han 
blir gira når medlemmer gir positiv feedback. 

Erlend tror mye av grunnen til at vi sporer av er på grunn av at vi ikke helt vet hva vi skal gjøre, og at vi har noen konkret 
arbeidsoppgave. Denne meningen deler David og Emil, og tror at det vil løse seg når gruppen kommer lenger ut i prosjektet og får 
formulert en problemstilling og delegert arbeidsoppgaver. Det kan lett bli litt frusterasjon når en ikke ikke har et klart mål på hva en 
skal gjøre. I følge Schatz, 2002, er det viktig å ha et klart formål med gruppen, en grunn til at den skal eksistere.

Gruppen under ett syns det var bra noen tok opp situasjonen, ikke bare fordi avsporing kan føre til lav effektivitet, men også fordi det 
er tydelig at folk har ulik tolkning av situasjoner. Mens Nina synes det virket som gruppen satt å sjøldigget, tolket Christoffer og 
Kristoffer at gruppemedlemmene egentlig bare ville oppmuntre hverandre siden vi ikke kjente hverandre så godt, og vi var ute etter å 
bygge et positivt miljø i gruppen. Dette understreker at det er viktig med lav terskel for å ta opp saker som enkelte kan tolke negativt 
uten at andre gjør det.

Det viktigste gruppen sitter igjen med denne situasjonen er at det burde være lav terskel for å ta opp saker siden folk kan tolke ting så 
ulikt, og vi burde ikke være for redd å pakke ting i for mye ``vennlighet'', slik at budskapet blir forvrengt. Angående selve avsporingen 
og at det blir mye prating uten noen notater så burde gruppen bli flinkere til å ta notater under diskusjonene, slik at om viktig poeng 
kommer opp blir de husket.

\section{Evaluering av aksjoner}
En lavere terskel for å ta opp slike temaer som dette har ført til mer åpenhet i gruppen. Gruppemedlemmene ser itterkant ut til å ha lettere for å komme med kommentarer, og slipper å sitte inne med ting hele dagen før de tørr å ta det opp.


